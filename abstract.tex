%----------------------------------------------------------------------------
% Abstract in hungarian
%----------------------------------------------------------------------------
\chapter*{Kivonat}

Jelen dokumentum egy a Budapesti Műszaki és Gazdaságtudományi Egyetem Irányítástechnika és Informatika Tanszékén készített diplomaterv \emph{,,Tekintetkövető rendszer fejlesztése képfeldolgozási alapokon''} témában. 

\bigskip

Diplomatervemben egy költséghatékony, képfeldolgozási módszerekre alapuló tekintetkövető rendszer kifejlesztését tűztem ki célul, ugyanis a tekintet követése számos felhasználási területen bizonyulhat hasznosnak. Első lépésben kutatási és gyakorlati területen vizsgáltam a rendszer felhasználhatóságát, annak érdekében, hogy megmutassam, nagy potenciál rejtőzik egy pontos és robusztus követési megoldás létrehozásában.

Ezután megvizsgáltam a Hough-transzformációt, a Viola--Jones objektumdetektort és a blob-alapú követést mint lehetséges megoldásokat, annak érdekében, hogy kiválaszthassam a leginkább megfelelő módszert a pupillakövetés megvalósításához, ami az optikai elvű követés alapja.

Blobok detektálására és analízisére alapozva létrehoztam egy működő tekintetkövető rendszert, amelyben képfeldolgozási célokra az OpenCV gépi látás könyvtárat, a felhasználói felület kialakításához pedig a Qt keretrendszert használtam fel.


\newpage

%----------------------------------------------------------------------------
% Abstract in english
%----------------------------------------------------------------------------
\chapter*{Abstract}

The current paper is a thesis from Department of Control Engineering and Information Technology at Budapest University of Technology and Economics in the topic of \emph{,,Development of a Gaze Tracking System Based on Image Processing''}. 

\bigskip

In my thesis research has been done on how to create an inexpensive gaze tracking system based on image processing methods. There are several fields of application where a gaze tracking system can be used. First I studied the two main fields --- research and practical --- to show the potential of an efficient and robust tracking solution.

Then I evaluated the Hough-transform, the Viola--Jones cascade classifier and blob-based methods. My objective was to find out which one is the best fit for solving the problem of pupil tracking --- the cornerstone of this area.

Based on blob analysis I've presented a solution for a working gaze tracking system using the OpenCV library for image processing, and the Qt framework for building the application's user interface. 

\vfill
%\clearpage~
