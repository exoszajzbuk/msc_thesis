%----------------------------------------------------------------------------
% Abstract in hungarian
%----------------------------------------------------------------------------
\chapter*{Kivonat}

Jelen dokumentum egy a Budapesti Műszaki és Gazdaságtudományi Egyetem Irányítástechnika és Informatika Tanszékén készített diplomaterv \emph{,,Tekintetkövető rendszer fejlesztése képfeldolgozási alapokon''} témában. 

Diplomatervemben egy költséghatékony tekintetkövető rendszer kifejlesztését tűztem ki célul képfeldolgozási módszerekre alapozva. Összegyűjtöttem, hogy milyen a rendszer használatával milyen lehetséges felhasználási területek jöhetnek szóba. Ezután megvizsgáltam a Hough-transzformációt, a Viola--Jones objektumdetektort és a blob-alapú követést mint lehetséges megoldásokat, annak érdekében, hogy kiválaszthassam a leginkább megfelelő módszert.




\newpage

%----------------------------------------------------------------------------
% Abstract in english
%----------------------------------------------------------------------------
\chapter*{Abstract}

The current paper is a thesis from Department of Control Engineering and Information Technology at Budapest University of Technology and Economics in the topic of \emph{,,Development of a Gaze Tracking System Based on Image Processing''}. 

\bigskip

In my thesis research has been done on how to create an inexpensive gaze tracking system using image processing methods.

There are several fields of application where a gaze tracking system can be used. First I TODOstudied the two main fields --- academic and practical --- to show the potential of an efficient and robust tracking solution.

Then I evaluated the Hough-transform, the Viola--Jones cascade classifier and blob-based methods. My objective was to find out which one is the best TODOfit for solving the problem of pupil tracking --- the cornerstone of this area.

Based on blob analysis I've presented a solution for a working gaze tracking system using the OpenCV library for image processing, and the Qt framework for building the application's user interface. 

\vfill
%\clearpage~
