%----------------------------------------------------------------------------
\appendix
%----------------------------------------------------------------------------
\chapter*{Függelék}
\setcounter{chapter}{6}  % a fofejezet-szamlalo az angol ABC 6. betuje (F) lesz
\setcounter{equation}{0} % a fofejezet-szamlalo az angol ABC 6. betuje (F) lesz
\numberwithin{equation}{section}
\numberwithin{figure}{section}
\numberwithin{lstlisting}{section}
\numberwithin{table}{section}
\setcounter{footnote}{0}

%,,,,,,,,,,,,,,,,,,,,,,,,,,,,,,,,,,,,,,,,,,,,,,,,,,,,,,,,,,,,,,,,,,,,,,,,,,,,
\section{Mellékletek}\label{sect:mellekletek}
%,,,,,,,,,,,,,,,,,,,,,,,,,,,,,,,,,,,,,,,,,,,,,,,,,,,,,,,,,,,,,,,,,,,,,,,,,,,,

Jelen dokumentum, a féléves munkát bemutató prezentáció, valamint az alkalmazás forráskódja, elérhetők az Interneten is, a \url{http://mozes.info/pupilmeasure/} címen.

Az alkalmazás forráskódja a \url{https://github.com/obrien/pupilmeasure} címen található GitHub\footnote{\url{http://github.com/}} projektben is elérhető és felhasználható.

%,,,,,,,,,,,,,,,,,,,,,,,,,,,,,,,,,,,,,,,,,,,,,,,,,,,,,,,,,,,,,,,,,,,,,,,,,,,,
\section{Telepítés}\label{sect:telepites}
%,,,,,,,,,,,,,,,,,,,,,,,,,,,,,,,,,,,,,,,,,,,,,,,,,,,,,,,,,,,,,,,,,,,,,,,,,,,,

Az alkalmazás fordítása az alábbi verziójú programokkal történt:

\begin{itemize}
\item g++ --  4.4.5
\item OpenCV -- 2.1.0
\item wxWidgets -- 2.8.10
\end{itemize}

A  2.1.0 verziójú \textit{OpenCV} könyvtár forráskódja az \url{http://opencv.willowgarage.com/wiki/} címről letölthető. Az ennél újabb verziók interfészébe nem került bele a működéshez elengedhetetlenül szükséges \texttt{cvSnakeImage} függvény, ezért az alkalmazás mindenképpen a fenti verziójú \textit{OpenCV} könyvtár meglétét igényli! A fordításhoz szükséges csomagok a használt disztribúció csomagkezelőjéből elérhetők az \listref{install} listában felsorolt (vagy ahhoz hasonló) néven.

\begin{lstlisting}[frame=single,float=!ht,caption=Az OpenCV fordításához szükséges csomagok telepítése,label=listing:install]
sudo apt-get install libgtk2.0-dev libunicap2-dev libucil2-dev libswscale-dev \\
libdc1394-22-dev libv4l-dev libxine-dev libavformat-dev libglut3-dev \\
libwxgtk2.8-dev
\end{lstlisting}

A \emph{Pupil Measure} alkalmazás fordítása ezek után a forráskód könyvtárában kiadott \texttt{make} paranccsal történhet. A lefordított bináris a \texttt{./pupilmeasure} parancs kiadásával indítható, de előtte az \texttt{LD\_LIBRARY\_PATH} környezeti változóba a \texttt{usr/local/lib} érték betöltendő!
