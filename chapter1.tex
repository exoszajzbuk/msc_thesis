%----------------------------------------------------------------------------
\chapter{Felhasználási lehetőségek}\label{sect:felhasznalas}
%----------------------------------------------------------------------------

A tekintet megfelelő minőségű és robusztus követésének számos gyakorlati felhasználása lehetséges. Elég csak a \emph{perceptuális} (észlelési), vagy \emph{kognitív} (megértési) kutatási területekre gondolni, ahol például az olvasás, vagy az alvás folyamatának vizsgálatánál bizonyulhat hasznosnak.

%Pupilla szöveg.

%\bigskip

%Felépítés.

%,,,,,,,,,,,,,,,,,,,,,,,,,,,,,,,,,,,,,,,,,,,,,,,,,,,,,,,,,,,,,,,,,,,,,,,,,,,,
\section{Kutatási felhasználás}\label{sect:tudomanyos}
%,,,,,,,,,,,,,,,,,,,,,,,,,,,,,,,,,,,,,,,,,,,,,,,,,,,,,,,,,,,,,,,,,,,,,,,,,,,,

%Szöveg.

%............................................................................
\subsection{Kognitív pszichológia}\label{sect:kognitiv}
%............................................................................

Olvasási folyamatok terén a szakkádok gyors és pontos követése válhat lehetővé, amely a kognitív folyamatok elemzésében komoly segítséget nyújthat. Alvásvizsgálat terén leginkább -- nevéből adódóan -- a REM (Rapid Eye Movement) fázis kötődik a szemmozgás követéséhez, ebben az alkalmazásban azonban értelemszerűen optikai elvű követés nem jöhet szóba.

%............................................................................
\subsection{Érzelemdetektálás}\label{sect:erzelem}
%............................................................................

A pupillaátmérő nem csak a fénymennyiség-változás hatására módosulhat. Érzelmi, izgalmi állapotok is előidézőik a változást, mint például félelem, idegesség vagy öröm \cite{altpszicho}. Ez a jelenség szintén potenciális felhasználási lehetőségeket rejt magában. Mivel a pupillareflex akaratlagosan nem koordinálható, állandó fénymennyiség mellett a pupillaméret változásának figyelésével detektálhatóvá válhatnak a fent említett érzelmi állapotok. Ehhez a változás mértékének és sebességének pontos mérése szükséges, ami azonban kellően nagy sebességű kamerával és elfogadható számítási teljesítményt nyújtó hardverrel kielégítő minőségben megtehető lehet. Az alkalmazás ráadásul nem feltétlenül igényli a szem közvetlen közelről (például fejre erősített kamerával) történő felvételét. Megfelelően nagy felbontású forrás esetén az arc-, majd szemrégió automatikus szegmentálása után a felismert zónát felhasználva, akár távolról is történhet a pupillareflex vizsgálata.

%............................................................................
\subsection{Orvosi felhasználás}\label{sect:orvosi_felh}
%............................................................................

Egyes betegségek is okozhatják a pupilla rendellenes méretét vagy viselkedését. Például a ,,miosis'', azaz a szem összehúzódása nemcsak a fent említett okokra vezethető vissza. Rendellenes összehúzódás alakulhat ki bizonyos patológiai állapotok, gyógyszerek, vagy mérgek hatására, sõt a mikrohullámú sugárzásnak kitett szervezet is produkálja ezt a tünetet. A ,,mydraisis'' (a pupilla tágulása) során ugyancsak nem megszokott viselkedés alakulhat ki bizonyos gyógyszerek vagy kábítószerek használatakor, de akár komoly fizikai trauma hatására is a normálisnál jelentősebb mértékű vagy időtartamú lehet a pupilla tágulata. A két szem eltérő méretű pupillája (az ,,anisocoria'') olyan betegségek meglétét jelezheti, mint a Horner-, vagy az Adie-szindróma \cite{altpszicho}. Orvosi szempontból is van tehát mit vizsgálni: a pupilla követésével egyes betegségek, állapotok felismerése, vagy alakulásuk megfigyelése laikus és orvos számára is automatizálható, megkönnyíthető lehet.

\bigskip

Továbbra is orvosi területen maradva a szemmozgás követése és regisztrálása \emph{ontoneurológiai} vizsgálatokban is szerepet kaphat. Az ilyen vizsgálat célja az egyensúlyszerv működésének megfigyelése és értékelése. Az összetett vizsgálat egyes fázisaiban a szemmozgások követése fontos információt hordoz az alany állapotáról, ugyanis a szemmozgató és az egyensúlyi információkat szállító idegpályák szoros kapcsolatban állnak egymással.

%,,,,,,,,,,,,,,,,,,,,,,,,,,,,,,,,,,,,,,,,,,,,,,,,,,,,,,,,,,,,,,,,,,,,,,,,,,,,
\section{Gyakorlati felhasználás}\label{sect:gyakorlati}
%,,,,,,,,,,,,,,,,,,,,,,,,,,,,,,,,,,,,,,,,,,,,,,,,,,,,,,,,,,,,,,,,,,,,,,,,,,,,

%Szöveg.

%............................................................................
\subsection{Webergonómia}\label{sect:webergonomia}
%............................................................................

Ugyancsak felkapott kutatási terület manapság a \emph{webergonómia} területe. Kellően pontos követéssel vizsgálható lehet, hogy a dizájnereknek mennyire sikerült a felhasználók igényeinek megfelelő felületet alkotniuk: könnyen eligazodnak-e rajta, esetleg idejük nagy részét a rossz tervezés következtében kaotikus bolyongással töltik.

%............................................................................
\subsection{Vezetésbiztonság}\label{sect:orvosi_felhasznalas}
%............................................................................

Vezetésbiztonsági alkalmazásban is elképzelhető lehet a pupillakövetés alkalmazása. A követés során mintegy járulékos információként mérhetjük a pislogások gyakoriságát és hosszát, ezzel felismerhetővé válhat a gépjárművezetés közben lankadó figyelem, és jelezhető, ha fennáll az elalvás veszélye. Az eljárás így hasznosnak bizonyulhat már meglévő elalvásdetektálási módszerek \cite{sleepdet} kiegészítéseként, tovább javítva azok megbízhatóságát.

%,,,,,,,,,,,,,,,,,,,,,,,,,,,,,,,,,,,,,,,,,,,,,,,,,,,,,,,,,,,,,,,,,,,,,,,,,,,,
%\section{Összefoglalás}\label{sect:felh_osszefoglalas}
%,,,,,,,,,,,,,,,,,,,,,,,,,,,,,,,,,,,,,,,,,,,,,,,,,,,,,,,,,,,,,,,,,,,,,,,,,,,,

%Összefoglalás.