%----------------------------------------------------------------------------
\chapter{Megvalósítás I.}\label{sect:megvalositas_1}
%----------------------------------------------------------------------------

Megismerkedve a szükséges elméleti alapokkal, valamint kiválasztva a felhasználandó szoftveres és hardveres technológiákat, elkezdhettem a tekintetkövetés megvalósítását. A rendszer fejlesztésének első fázisában nem kezdtem bele rögtön komoly architektúra vagy grafikus felhasználói felület tervezésébe. Mielőtt ugyanis ezekre sor kerülhet, fontos döntések meghozatalára, alap algoritmusok fejlesztésére, majd a helyes működés igazolására van szükség. Ezekről a döntésekről, algoritmusokról és validációról szól dolgozatom \sectref{megvalositas_1} fejezete.

\bigskip

%Szöveg.

%,,,,,,,,,,,,,,,,,,,,,,,,,,,,,,,,,,,,,,,,,,,,,,,,,,,,,,,,,,,,,,,,,,,,,,,,,,,,
\section{Módszerek összehasonlítása}\label{sect:modsz_osszehasonlitas}
%,,,,,,,,,,,,,,,,,,,,,,,,,,,,,,,,,,,,,,,,,,,,,,,,,,,,,,,,,,,,,,,,,,,,,,,,,,,,

\texttt{+++ miert BLOB-os keresest valasztottam Hough vagy Optical flow helyett? +++}

%,,,,,,,,,,,,,,,,,,,,,,,,,,,,,,,,,,,,,,,,,,,,,,,,,,,,,,,,,,,,,,,,,,,,,,,,,,,,
\section{Feldolgozási folyamat}\label{sect:feld_folyamat}
%,,,,,,,,,,,,,,,,,,,,,,,,,,,,,,,,,,,,,,,,,,,,,,,,,,,,,,,,,,,,,,,,,,,,,,,,,,,,

A blob alapú követést választva ki kellett dolgoznom a tekintetkövetés működéséhez szükséges képfeldolgozási illetve számítási folyamatot. Két feladatra kellett robusztus megoldást találnom: a pupilla követésére, valamint a pupilla-pozíció képernyő-pozícióba történő leképezésére.

Mindkét feladat önmagában van annyira összetett, hogy érdemes őket kitüntetett figyelemmel kezelni. 

%............................................................................
\subsection{Pupillakövetés}\label{sect:pupillakov}
%............................................................................

A pupillakövetés megbízható működéséhez meg kellett találnom azon elemi képfeldolgozási műveletek megfelelő sorrendjét és paraméterezését, amelyekkel lehetővé vált a pupilla azonosítása. A pontos folyamat kialakítása nagyrészt intuíciók alapján, a részmegoldások folyamatos tesztelésével történt.

Az előfeldolgozási műveleteket az OpenCV készen kínálja, azonban a függvények száma és paraméterezési lehetőségei miatt rengeteg alternatíva jöhetett szóba. A szakasz további részében a kiválasztott képfeldolgozási műveleteket szeretném dokumentálni, a folyamat egészének tekintetében lényegtelen részletek mellőzésével. A felhasznált függvények pontos paraméterezése megtalálható az \sectref{parameterezes} függelékben.

\bigskip

A szemrégió képét közvetítő webkamera a gyári orientációjához képest fejjel lefelé került felfüggesztésre. A könnyebben értelmezhető megjelenítés érdekében ezért \emph{első lépésként} \textbf{megtükröztem} a képet a vízszintes középtengely mentén a \texttt{cv::flip} függvény használatával. Jegyezzük meg azonban, hogy ez a lépés csak azért szükséges, hogy a felhasználó számára emészthetőbb, ,,álló'' képet tudjunk megjeleníteni az alkalmazás felületén. A kép tükrözött voltát a feldolgozás során végig figyelembe véve ez a lépés akár el is hagyható!

A pupillakövetés szempontjából a színinformációnak nincs jelentősége, ezért a \emph{második lépés} a kép \textbf{szürkeárnyalatossá konvertálása} a \texttt{cv::cvtColor} metódus használatával.

A \emph{harmadik lépésben} \textbf{hisztogram-kiegyenlítést} végeztem a képen az OpenCV \texttt{cv::equalizeHist} függvénye használatával. A kiegyenlítés eredményeképp a kép minden megvilágítási körülmény között kellően kontrasztos lesz, megkönnyítve ezzel a pupilla detektálását.

TODO: hisztogram+kuszob

A szürkeárnyalatos képet ezek után a \emph{negyedik lépésben} \textbf{küszöbözni} kellett, ehhez a \texttt{cv::threshold} függvény használatára volt szükség. A pontos küszöbérték beállítása általában nehéz feladat, minden megvilágításra működő univerzális érték megtalálása pedig sokszor nem is lehetséges. A \sectref{infracam_mod} szakaszban részletezett infra-fényforrás megvalósításával azonban abba a szerencsés helyzetbe kerültem, hogy a külső megvilágítás hatását gyakorlatilag sikerült teljesen kiküszöbölni: a tesztelés során végül meghatározott ideális küszöbérték a verőfényes napsütéstől kezdve a vaksötét szobáig minden esetben megfelelően binarizálja a szürkeárnyalatos képet.

\bigskip

A fenti négy lépés elég a kamerakép előfeldolgozásához, az \emph{ötödik lépéstől} kezdve a tényleges felismerés megvalósítása következhet. Az előző, \sectref{modsz_osszehasonlitas} szakaszban kifejtettem, hogy a feldolgozás sebességét döntően befolyásolja a pixelszinten végzett feldolgozási lépések komplexitása és száma. A előfeldolgozás végén a kép még meglehetősen nagy zajjal terhelt: látható a pupilla viszonylag nagy, ovális foltja, azonban például a szempillák foltja zavaró hatásként jelentkezik.

Sebesség szempontjából azonban sokkal jobban járunk, ha nem is próbáljuk az előfeldolgozási lépések után hátramaradó zajt pixelszintű eljárásokkal kiszűrni (pl. bináris morfológia). Ehelyett az előfeldolgozott képre rögtön ráereszthetünk egy hierarchikus \textbf{kontúrkeresési eljárást} a \texttt{cv::findContours} függvény meghívásával. A további lépések során elegendő a hierarchia legfelső szintjét használnunk: mivel a megtalálni kívánt pupilla konvex, a blobok belsejében található lyukak elhelyezkedése nem szolgál többletinformációval.

\emph{Hatodik lépésben} meg is szabadulhatunk a \textbf{túl kis méretű objektumoktól}, mivel ezek nagy valószínűséggel csak mint zaj vannak jelen a képen. Egy kontúrpontjaival megadott objektum területét a \texttt{cv::contourArea} függvény segítségével határozhatjuk meg, ez alapján már kiszűrhetők a ,,túl kicsi'' objektumok.

A kisméretű zajok eltávolítása után jó eséllyel csak néhány, nagyobb területű blobunk marad, köztük a megtalálni kívánt pupilla. A fennmaradó blobok között jó szűrési feltételként adja magát a blobok \textbf{körkörösségének} vizsgálata. A \emph{hetedik lépésben} ezért a \sectref{blob_analizis} szakaszban ismertetett módon kiszámolom a foltok körkörösségi faktorát, majd ezek közül a legerősebbet kiválasztva nagyon jó százalékban sikerül a pupilla azonosítása.

TODO: blobs + ellipse

Nincs más hátra, mint a \emph{nyolcadik lépésben} \textbf{ellipszist illeszteni} a pupilla kontúrjára, ehhez használható a \texttt{cv::fitEllipse} függvény. A pupillát a ráillesztett ellipszis paramétereivel lehet jellemezni: középpontjával, kis- és nagytengelyének hosszával, valamint nagytengelyének irányszögével.

\bigskip

A pupillakövetésnél gondolni kell azokra az esetekre is, amikor a követés fizikailag nem megvalósítható. Csukott szemnél a követés nyilvánvalóan lehetetlen, de az pislogási reflex okozta anomáliákat jó lenne kiszűrni. A detektálás közben felismerhető, ha egyáltalán nincs értékes pupilla-jelölt a blobok között. Abban az esetben, ha a fent vázolt algoritmus nem találja a pupillát, \textbf{öt egymást követő sikertelen felismerésig} még megtartja az utoljára detektált pupillapozíciót. Ezzel a módszerrel a pislogás okozta rövid kiesések szinte teljesen kiküszöbölhetők, és a valóban csukott szem detektálása is csak néhány századmásodperces késleltetést szenved.

%............................................................................
\subsection{Kalibráció, leképezés}\label{sect:kalibracio}
%............................................................................

A pupillakövetés mellett a másik, összetettebb algoritmus használatát igénylő feladat a \textbf{tekintet kalibrációja} volt, annak érdekében, hogy a kameraképen detektált pupillapozíció és a képernyőn nézett pont között kapcsolatot teremthessek.

A kalibráció során a képernyő négy sarkában rögzítem a pupilla pozícióját, majd minden képkockán ezek között a pontok között interpolálva számítom ki a képernyőn nézett pont koordinátáit.

\bigskip

A klasszikus \textbf{bilineáris interpoláció} estén az egységnégyzet $u = [0, 1]$ és $v = [0, 1]$ koordinátáit szeretnénk egy tetszőleges négyszögbe leképezni.

TODO: interpolacio

A TODO ábrán látható módon definiáljuk a négyszöget $A$, $B$, $C$ és $D$ sarkaival. Belátható, hogy a $P$ pont koordinátáit a következő módon kaphatjuk meg:

\begin{itemize}
  \item $E$ pont lineáris interpolációval adódik $A$ és $B$ között
  \item $F$ pont lineáris interpolációval adódik $C$ és $D$ között
  \item a $P$ pont szintén lineáris interpolációval adódik $E$ és $F$ között
\end{itemize}

A leképezési feladat megoldása során a fenti számítást kell ,,visszafelé'' elvégezni: a sarokpontok (pupillapozíciók a képernyő sarkain mérve) mellett szintén adott $P$ ponthoz (a pupilla aktuális képkockán vett pozíciója) kell meghatározni annak $u$ és $v$ koordinátáit. Az $u$ és $v$ koordináták, valamint a képernyő szélességének és magasságának ($w$ és $h$) ismeretében meghatározhatók az éppen fókuszált pont képernyő-koordinátái, $l$ és $m$.

A felírandó rendszer ugyan nem lineáris, de egyszerűen, analitikusan megoldható. Az $u$ koordináta az

\begin{align}\label{eq:calib_1}
u &= \frac{-b - \sqrt{b^2 - 4 \cdot a \cdot c}}{2 \cdot a}
\end{align}

másodfokú egyenlet megoldásaként számítható, ahol az együtthatókat a következő kifejezések adják meg:

\begin{align}\label{eq:calib_2}
a &= (B_y - A_y) \cdot (C_x - D_x) - (B_x - A_x) \cdot (C_y - D_y) \nonumber \\
b &= (A_y - P_y) \cdot (C_x - D_x) + (B_y - A_y) \cdot (D_x - P_x) - \nonumber \\ 
  & \qquad - (A_x - P_x) \cdot (C_y - D_y) - (B_x - A_x) \cdot (D_y - P_y) \nonumber \\
c &= (A_y - P_y) \cdot (D_x - P_x) - (A_x - P_x) \cdot (D_y - P_y)
\end{align}

Az $u$ koordináta ismeretében már meghatározhatók $E$ és $F$ pontok, a

\begin{align}\label{eq:calib_3}
E &= A + (B-A) \cdot u \nonumber \\
F &= D + (C-D) \cdot u
\end{align}

kifejezések kiszámításával. A hiányzó $v$ koordinátát ezek után a 

\begin{align}\label{eq:calib_4}
v &= (P_x - E_x) / (F_x - E_x)
\end{align}

képlettel határozhatjuk meg. Az egységnégyzetet már csak a megfelelő méretre kell skálázni: az $l$ és $m$ képernyő-koordináták a következő triviális összefüggésekkel számíthatók.

\begin{align}\label{eq:calib_5}
l &= u \cdot w \nonumber \\
m &= v \cdot h
\end{align}

%............................................................................
\section{Validáció}\label{sect:validacio}
%............................................................................

Az alapvető algoritmusok önálló implementációját követően, de még az architektúra kialakítása, valamint a felhasználói felületek fejlesztésének megkezdése előtt szerettem volna igazolni, hogy a kifejlesztett pupillakövetési módszer valóban alkalmas a rá szabott feladat elvégzésére.

A validáció során méréseket végeztem a pupillakövető algoritmussal: a kijelző teljes területén sorban egy adott oldalhosszúságú négyzetháló pontjaira fókuszálva vettem fel a felismert ellipszisek paramétereit. A méréseket egy 1280$\times$1024 képpont felbontású monitorral végeztem el, a fej pozícióját álltámasszal rögzítve (a mérési elrendezés pontos paraméterei a \sectref{parameterezes} függelékben megtalálhatóak). A mérés során használt négyzetháló méretét és az ebből adódóan rögzített képpontok számát a \tabref{meresi_pontok} táblázat tartalmazza.

\begin{table}[ht]
	\centering
	\caption{A validáció során rögzített mérések adatai.} \label{tab:meresi_pontok}
	\begin{tabular}{ c | c }
	rács mérete & rögzített pontok száma \\ \hline \hline
	100$\times$100 képpont & 143 db \\
	75$\times$75 képpont & 252 db \\
	50$\times$50 képpont & 546 db \\
	\end{tabular}
\end{table}

Mindhárom rácsméret esetén a rácspontokon balról jobbra, fentről lefelé haladva minden mérési pontban a következő adatokat rögzítettem a tesztpontra és a felismert ellipszisre vonatkozóan:

\begin{itemize}
  \item a tesztpont \textbf{képernyő-koordinátái} -- $T_x$ és $T_y$
  \item az ellipszis \textbf{súlypontjának} pozíciója -- $C_x$ és $C_y$
  \item az ellipszis \textbf{kis- és nagytengelyhossza} -- $k$ és $l$
  \item az ellipszis \textbf{irányszöge} -- $\varphi$
\end{itemize}

\emph{Első lépésként} ábrázoltam a felismert ellipszis súlypontjának (középpontjának) koordinátáit a tesztpont koordinátáinak függvényében. Az eredmény a TODO ábrán látható, és a többnyire szabályos mintázat megjelenése jó kilátásokkal kecsegtetett a későbbi mérések, valamint a remélhetőleg helyes működés tekintetében is.

TODO

A \emph{második lépésben} a volt a célom, hogy meghatározzam, az ellipszis paraméterei közül melyek azok, amelyek a legszignifikánsabban vesznek részt a pupillapozíció és a képernyő-koordináta közötti kapcsolatban. A paraméterek ,,fontossági'' értékeit \textbf{korrelációs együtthatók} számításával vizsgáltam a \tabref{korrel} táblázatban megadott bemenetekkel és eredményekkel. 

\begin{table}[ht]
	\centering
	\caption{Korrelációs együtthatók.} \label{tab:korrel}
	\begin{tabular}{ c | c }
	rács mérete & rögzített pontok száma \\ \hline \hline
	100$\times$100 képpont & 143 db \\
	75$\times$75 képpont & 252 db \\
	50$\times$50 képpont & 546 db \\
	\end{tabular}
\end{table}

Nem teljesen váratlan módon a legnagyobb korreláció minden esetben az ellipszis súlypontjának koordinátái és a tesztpont koordinátái között mutatkozott. A $0,\!9$ feletti korrelációs együtthatók alapján bátran kijelenthető, hogy lehetséges csak a súlypont koordinátáit felhasználni a leképezés során.

Látható azonban, hogy a többi paraméter esetében sem túlzottan kicsik a korrelációs együtthatók: a további fejlesztések során ezen paramétereket járulékos információként felhasználva a tekintetkövetés pontossága növelhető lehet.

%,,,,,,,,,,,,,,,,,,,,,,,,,,,,,,,,,,,,,,,,,,,,,,,,,,,,,,,,,,,,,,,,,,,,,,,,,,,,
\section{Demonstráció}\label{sect:demonstracio}
%,,,,,,,,,,,,,,,,,,,,,,,,,,,,,,,,,,,,,,,,,,,,,,,,,,,,,,,,,,,,,,,,,,,,,,,,,,,,

A számszerűsített validáció mellett -- még mindig csak a lehető legegyszerűbben implementált pupillakövetési és leképezési algoritmusok felhasználásával -- szükségét éreztem annak is, hogy megmutassam: a rendszer maximálisan valós problémák megoldására is alkalmas lehet.

Munkám \sectref{felhasznalas} fejezetében többek között a pszichológiai és a webergonómiai felhasználási lehetőségekről ejtettem szót. A szakaszban további részében lássunk tehát e két témában egy-egy egyszerű kísérletet, a rendszer gyakorlati használhatóságát igazolandó. 

%............................................................................
\subsection{Pszichológiai bemutató}\label{sect:pszicho}
%............................................................................

A rendszer magasabb szintű működését elsőként úgy próbáltam demonstrálni, hogy végrehajtottam Alfred L. Yarbus orosz pszichológus 1967-es tanulmánya egy részletét. A kísérletben a kutatók azt bizonyították be, hogy a tesztalanyoknak előzetesen különböző kérdéseket feltéve, a kérdések jelentősen befolyásolták egy kép részleteinek vizsgálatát ahhoz képest, ha csak ,,szabadon'' nézegették azt. A teszteredményeket összefoglaló kép -- mint az egyik első a témával foglalkozó eredmény -- jól ismert a tekintetkövetéssel foglalkozók körében, ez látható a \figref{yarbus} ábrán.

\begin{figure}[!ht]
\centering
\includegraphics[width=140mm, keepaspectratio]{figures/yarbus.jpg}
\caption{Yarbus '67-es kísérletének eredménye}
\label{fig:yarbus}
\end{figure}

Látható, hogy a kísérlet végrehajtásához szükség volt a tekintet követésére. Ekkor még a kornak megfelelően nem álltak rendelkezésre kifinomult módszerek: az alanyokat egy meglehetősen kényelmetlen acélszerkezethez rögzítve vizsgálták. A bemutató alkalmazásomban arra kerestem a választ, hogy lehetséges-e hasonló méréseket elvégezni az általam fejlesztett rendszerrel.

\bigskip

Az általam tesztelt alanyoknak a következő kérdésekre kellett válaszolniuk a tesztkép (Repin: Váratlan utazó) rövid vizsgálata után. A vizsgálatot minden kérdésben 200 beérkezett érvényes mérési pontig folytattam, ez a felhasznált webkamera, illetve a feldolgozási sebesség mellett nagyjából kérdésenként 15--20 másodpercet vett igénybe.

\begin{enumerate}
 \item szabad nézelődés
 \item ,,Milyen anyagi körülmények között él a család?''
 \item ,,Adja meg az egyes szereplők életkorát!''
 \item ,,Mit csinálhattak a szereplők, mielőtt az utazó betoppant?''
 \item ,,Milyen ruhát viselnek a kép szereplői?''
 \item ,,Próbáljon megjegyezni minél több strukturális részletet (személyek, tárgyak pozíciója)!''
 \item ,,Mennyi ideig lehetett távol az utazó a családtól?''
\end{enumerate}

A kérdésekre nem volt jó, vagy rossz válasz, a kísérlet minden alanynál csak a feladatnak megfelelő figyelmi területek változását vizsgálja. A vizsgálatom eredménye a \figref{eredmeny} ábrán követhető.

\begin{figure}[!ht]
\centering
\includegraphics[width=140mm, keepaspectratio]{figures/yarbus_eredmeny.png}
\caption{Eredmények a saját rendszerrel}
\label{fig:eredmeny}
\end{figure}

Látható -- bár ez nem volt feltétel -- hogy a mérési eredmények ezen alany esetén meglehetősen jól fedik Yarbus tesztalanyának eredményeit. Az viszont mindenképpen kijelenthető, hogy szignifikáns különbség van az ábra első (szabad nézelődés), valamint többi része között. Nem célom a kísérlet pszichológiai eredményeit értékelni, az viszont bizonyos, hogy a rendszer hasonló kísérletek elvégzését támogatja.

%,,,,,,,,,,,,,,,,,,,,,,,,,,,,,,,,,,,,,,,,,,,,,,,,,,,,,,,,,,,,,,,,,,,,,,,,,,,,
\subsection{Webergonómiai bemutató}\label{sect:web}
%,,,,,,,,,,,,,,,,,,,,,,,,,,,,,,,,,,,,,,,,,,,,,,,,,,,,,,,,,,,,,,,,,,,,,,,,,,,,

Ahogyan dolgozatom első fejezetében már említettem, a tekintet követése a webergonómia területén is fontos kísérletek elvégzésére ad lehetőséget. A rendszer ilyen irányú képességeit demonstrálandó, elkészítettem az pszichológiai kísérlet adatait hőtérképen (heatmap) megjelenítő funkciót is, ennek eredménye egy tesztképre a \figref{heatmap} ábrán látható.

\begin{figure}[!ht]
\centering
\includegraphics[width=100mm, keepaspectratio]{figures/heatmap.jpg}
\caption{Hőtérképes megjelenítés a pszichológiai teszt egy adathalmazára (,,Adja meg a szereplők életkorát!'')}
\label{fig:heatmap}
\end{figure}

Minden vizsgálatban célszerű a lehető leginkább informatív megjelenítést választani a rendelkezésre álló adathalmaz alapján. A képen egyre pirosabb színnel jelöltem a kép frekventáltan vizsgált területeit, például egy webergonómiai vizsgálatban az adatok effajta vizualizációja lehet a legcélszerűbb.

\bigskip

\texttt{+++ meg 1-2 bekezdes +++}

%,,,,,,,,,,,,,,,,,,,,,,,,,,,,,,,,,,,,,,,,,,,,,,,,,,,,,,,,,,,,,,,,,,,,,,,,,,,,
\section{Összefoglalás}\label{sect:kiserlet_osszefoglalas}
%,,,,,,,,,,,,,,,,,,,,,,,,,,,,,,,,,,,,,,,,,,,,,,,,,,,,,,,,,,,,,,,,,,,,,,,,,,,,

\texttt{+++ osszefoglalas a kiserletekrol +++}