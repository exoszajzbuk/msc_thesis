%----------------------------------------------------------------------------
\chapter*{Bevezető}
%----------------------------------------------------------------------------

\begin{verse}
\begin{flushright}
\emph{Hová merűlt el szép szemed világa? \\
Mi az, mit kétes távolban keres?} \\
\dots
\end{flushright}
\end{verse}

Tekintetkövető rendszer fejlesztése összetett feladat. A probléma mind magas, mind alacsony absztrakciós szintről megközelítve megfelelő szakmai felkészültséget követel meg: a rendszer megtervezéséhez, összeállításához és implementációjához olyan önálló mérnöki munka szükséges, ami a ideálissá tette számomra a feladatot diplomatervem témájaként.

\bigskip

A körülöttünk lévő világ megismerésében nyilvánvalóan alapvető szerepet tölt be a látás, ezen belül is a figyelmünk célzott irányítása. A vizuális feldolgozás meglétét és működését a legtöbbször olyannyira készpénznek vehetjük, hogy hajlamosak lehetünk elsiklani a tény fölött, hogy a megértés egyik első lépcsőjeként jelen lévő tekintet követése és analízise milyen mély jelentést hordozhat, legyen szó akár pszichológiai, akár fiziológiai vonatkozásokról.

A tekintet információgazdagságának köszönhetően reményeim szerint a rendszer -- valamint a fejlesztése során megszerzett tapasztalat -- remélhetőleg sok hasznos és érdekes kutatási vagy ipari felhasználásban bizonyulhat alkalmazhatónak.

\bigskip

Diplomatervem \sectref{felhasznalas} fejezetében a tekintetkövető rendszer lehetséges felhasználási területeiről nyújtok rövid áttekintést, a \sectref{tekintetkovetes} fejezetben pedig magát a tekintetkövetést veszem górcső alá a vonatkozó szakirodalomban fellelhető módszerek és a követendő szemmozgások összefoglalásával. Dolgozatom \sectref{elmeleti_alapok} fejezetében a tekintetkövetéshez felhasznált tudományos módszerek elméleti alapjait tárgyalom.

Gyakorlatiasabb vizekre evezve: a \sectref{technologia} fejezet technológiai áttekintésként szolgál, azaz bemutatom benne a fejlesztés során felhasznált szoftver- és hardverkomponenseket. Az \sectref{megvalositas_1} és \sectref{megvalositas_2} fejezetek pedig már kifejezetten a rendszer implementációjáról szólnak. A szétválasztást az indokolta, hogy amíg az \sectref{megvalositas_1} fejezetben magasabb szinten, a felesleges részletek mellőzésével mutatom be a legfontosabb feldolgozási algoritmusokat és igazolom a rendszer használhatóságát, addig a \sectref{megvalositas_2} fejezet első fele kimondottan az alkalmazás architektúrájával és az implementáció részleteivel foglalkozik.

Végezetül a \sectref{megvalositas_2} fejezet második felében dokumentálom az elkészült alkalmazás felhasználói felületét és használati eseteit, valamint összefoglalom és értékelem a munka során gyűjtött tapasztalataimat.