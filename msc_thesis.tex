\documentclass[12pt,a4paper,oneside]{report}             % Single-side
%\documentclass[11pt,a4paper,twoside,openright]{report}  % Duplex

%\PassOptionsToPackage{chapternumber=Huordinal}{magyar.ldf}
\usepackage{t1enc}
\usepackage[utf8]{inputenc}
\usepackage{amsmath}
\usepackage{amssymb}
\usepackage{enumerate}
%\usepackage[thmmarks]{ntheorem}
\usepackage{graphics}
\usepackage{epsfig}
\usepackage{listings}
\usepackage{color}
\usepackage{fancyhdr}
\usepackage{lastpage}
\usepackage{anysize}
\usepackage{amsthm}
\usepackage[magyar]{babel}
%\usepackage{indentfirst}
%\usepackage{natbib}
\usepackage{sectsty}
\usepackage{setspace}  % Ettol a tablazatok, abrak, labjegyzetek maradnak 1-es sorkozzel!
\usepackage[hang,compatibility=false]{caption}
%\usepackage{hyperref}
\usepackage{bookmark}
\usepackage{wrapfig}

\newtheorem{definition}{Def}

%--------------------------------------------------------------------------------------
% Main variables
%--------------------------------------------------------------------------------------
\newcommand{\mitauthor}{Kovács Balázs}
\newcommand{\mitadvisor}{Kertész Zsolt}
\newcommand{\mittitle}{Tekintetkövető rendszer fejlesztése képfeldolgozási alapokon}

%--------------------------------------------------------------------------------------
% Custom hyphenations
%--------------------------------------------------------------------------------------
\hyphenation{OpenCV}
\hyphenation{computer}
\hyphenation{vision}
\hyphenation{Qt Creator}
\hyphenation{session}
\hyphenation{Linux}
\hyphenation{Ubuntu}
\hyphenation{toolkit}
\hyphenation{wxWidgets}
\hyphenation{AdaBoost}
\hyphenation{screenshot}

%--------------------------------------------------------------------------------------
% Redefine original \chapter and \section commands
%--------------------------------------------------------------------------------------
\let\oldchap=\chapter
\renewcommand*{\chapter}{\secdef{\chapternostar}{\chapterstar}}
%\newcommand\chapterstar[1]{\oldchap*{#1}\markboth{~}{#1}\addcontentsline{toc}{chapter}{#1}}
\newcommand\chapterstar[1]{\phantomsection\addcontentsline{toc}{chapter}{#1}\oldchap*{#1}\markboth{~}{#1}}
%The first argument to \chapter is optional, hence the need for "[]"
%in the following definition. However, \secdef duplicates the mandatory
%argument if no optional argument was given. So we'll always have that
%argument and the default doesn't matter.
%\newcommand\chapternostar[2][]{\oldchap[#1]{#2}\markboth{\thechapter. #1}{\thechapter. #1}}
\newcommand\chapternostar[2][]{\phantomsection\oldchap[#1]{#2}\markboth{\thechapter. #1}{\thechapter. #1}}

\let\oldsect=\section
\renewcommand*{\section}{\secdef{\sectionnostar}{\sectionstar}}
%\newcommand\sectionstar[1]{\oldsect{#1}\markright{\thesection. #1}}
\newcommand\sectionstar[1]{\phantomsection\oldsect{#1}\markright{\thesection. #1}}
%\newcommand\sectionnostar[2][]{\oldsect[#1]{#2}\markright{\thesection. #1}}
\newcommand\sectionnostar[2][]{\phantomsection\oldsect[#1]{#2}\markright{\thesection. #1}}

\let\oldsubsect=\subsection
\renewcommand*{\subsection}{\secdef{\subsectionnostar}{\subsectionstar}}
\newcommand\subsectionstar[1]{\phantomsection\oldsubsect{#1}\markright{\thesection. #1}}
\newcommand\subsectionnostar[2][]{\phantomsection\oldsubsect[#1]{#2}\markright{\thesection. #1}}

\let\oldsubsubsect=\subsubsection
\renewcommand*{\subsubsection}{\secdef{\subsubsectionnostar}{\subsubsectionstar}}
\newcommand\subsubsectionstar[1]{\phantomsection\oldsubsubsect{#1}\markright{\thesection. #1}}
\newcommand\subsubsectionnostar[2][]{\phantomsection\oldsubsubsect[#1]{#2}\markright{\thesection. #1}}

%--------------------------------------------------------------------------------------
% Page layout setup
%--------------------------------------------------------------------------------------
% we need to redefine the pagestyle plain
% another possibility is to use the body of this command without \fancypagestyle
% and use \pagestyle{fancy} but in that case the special pages
% (like the ToC, the References, and the Chapter pages)remain in plane style
\fancypagestyle{plain}
{
	\fancyhead[R]{}
	\fancyhead[CO]{\bfseries\footnotesize\nouppercase{\rightmark}}
	\fancyhead[C]{\bfseries\footnotesize\nouppercase{\leftmark}}
	\fancyhead[L]{}
	\fancyfoot[L,RO]{}%\thepage}%{\thepage\pageref{LastPage}}
	\fancyfoot[C]{\thepage}
	\fancyfoot[LO,R]{}
	\renewcommand{\headrulewidth}{0.4pt}
%	\renewcommand{\footrulewidth}{0.4pt}
}

\pagestyle{plain}
\setlength{\headheight}{15pt}
%\setlength{\parindent}{0pt} % áttekinthetőbb, angol nyelvű dokumentumokban jellemző
%\setlength{\parskip}{8pt plus 3pt minus 3pt} % áttekinthetőbb, angol nyelvű dokumentumokban jellemző
\setlength{\parindent}{12pt} % magyar nyelvű dokumentumokban jellemző
\setlength{\parskip}{0pt}    % magyar nyelvű dokumentumokban jellemző

\marginsize{35mm}{25mm}{15mm}{15mm} % anysize package
\setcounter{secnumdepth}{0}
%\setcitestyle{authoryear, round, comma, aysep={;}, yysep={,}, notesep={, }}
\sectionfont{\Large\upshape\bfseries}
\subsectionfont{\large\upshape\bfseries}

\setcounter{secnumdepth}{2}
\singlespacing
\frenchspacing

%--------------------------------------------------------------------------------------
%	Setup hyperref package
%--------------------------------------------------------------------------------------
\hypersetup{
    %bookmarks=true,           % show bookmarks bar?
    bookmarksnumbered=true,   % set numbering
    unicode=true,             % non-Latin characters in Acrobat’s bookmarks
    pdftitle={\mittitle},     % title
    pdfauthor={\mitauthor},   % author
    pdfsubject={diplomaterv},   % subject of the document
    pdfcreator={\mitauthor},  % creator of the document
    pdfproducer={Kovács Balázs},   % producer of the document
    pdfkeywords={tekintetkövetés, gaze tracking, képfeldolgozás, image processing},   % list of keywords
    pdfnewwindow=true,        % links in new window
    colorlinks=true,          % false: boxed links; true: colored links
    linkcolor=black,          % color of internal links
    citecolor=black,          % color of links to bibliography
    filecolor=black,          % color of file links
    urlcolor=black            % color of external links
}
%--------------------------------------------------------------------------------------
%	Some new commands and declarations
%--------------------------------------------------------------------------------------
\renewcommand{\captionfont}{\small\itshape}
\renewcommand{\captionlabelfont}{\small\upshape\bfseries}
\newcommand{\code}[1]{{\upshape\ttfamily\scriptsize\indent #1}}
%\newcommand{\url}[1]{{\upshape\ttfamily\normalsize #1}}

% define references
\newcommand{\figref}[1]{\ref{fig:#1}.}
\renewcommand{\eqref}[1]{(\ref{eq:#1})}
\newcommand{\listref}[1]{\ref{listing:#1}.}
\newcommand{\sectref}[1]{\ref{sect:#1}.}
\newcommand{\tabref}[1]{\ref{tab:#1}.}

\DeclareMathOperator*{\argmax}{arg\,max}
%\DeclareMathOperator*[1]{\floor}{arg\,max}
\DeclareMathOperator{\sign}{sgn}
\DeclareMathOperator{\rot}{rot}
\definecolor{lightgray}{rgb}{0.95,0.95,0.95}

\author{\mitauthor, \mitauthork}
\title{\mittitle}
\includeonly{
	preliminaries,%
	statement,
	abstract,
	introduction,%
	chapter1,%
	chapter2,%
	chapter3,%
	chapter4,%
	chapter5,%
	chapter6,%
	assessment,%
	acknowledgement,%
	appendices,%
}
%--------------------------------------------------------------------------------------
%	Setup captions
%--------------------------------------------------------------------------------------
\captionsetup[figure]{
%labelsep=none,
%font={footnotesize,it},
%justification=justified,
width=.75\textwidth,
aboveskip=10pt}

\renewcommand{\captionlabelfont}{\small\bf}
\renewcommand{\captionfont}{\footnotesize\it}

%--------------------------------------------------------------------------------------
% Table of contents and the main text
%--------------------------------------------------------------------------------------
\begin{document}

\singlespacing
%--------------------------------------------------------------------------------------
%	The title page
%--------------------------------------------------------------------------------------
\begin{titlepage}
\begin{center}

\pdfbookmark[0]{Címoldal}{titlepage}

\includegraphics[width=100mm,keepaspectratio]{figures/BMElogo.png}\\
\textsc{Budapesti Műszaki és Gazdaságtudományi Egyetem}\\
\textsc{Villamosmérnöki és Informatikai Kar}\\
\textsc{Irányítástechnika és Informatika Tanszék}\\[5cm]

\vspace{0.4cm}
{\huge \bfseries \mittitle}\\[0.8cm]
\vspace{0.5cm}
\textsc{\Large Diplomaterv}\\[4cm]

\begin{tabular}{cc}
 \makebox[7cm]{\emph{Készítette}} & \makebox[7cm]{\emph{Konzulens}} \\
 \makebox[7cm]{\mitauthor} & \makebox[7cm]{\mitadvisor}
\end{tabular}

\vfill
{\large \today}
\end{center}
\end{titlepage}

%--------------------------------------------------------------------------------------
% Set up listings
%--------------------------------------------------------------------------------------
\lstset{
	basicstyle=\scriptsize\ttfamily, % print whole listing small
	keywordstyle=\color{black}\bfseries\underbar, % underlined bold black keywords
	identifierstyle=, 					% nothing happens
	commentstyle=\color{white}, % white comments
	stringstyle=\scriptsize\sffamily, 			% typewriter type for strings
	showstringspaces=false,     % no special string spaces
	aboveskip=3pt,
	belowskip=3pt,
	columns=fixed,
	backgroundcolor=\color{lightgray},
} 		
\def\lstlistingname{lista}

%\newpage\thispagestyle{empty} % an empty page

\pagenumbering{Roman}
\tableofcontents\vfill\markboth{Tartalomjegyzék}{~}

\include{statement}
%----------------------------------------------------------------------------
% Abstract in hungarian
%----------------------------------------------------------------------------
\chapter*{Kivonat}

Jelen dokumentum egy a Budapesti Műszaki és Gazdaságtudományi Egyetem Irányítástechnika és Informatika Tanszékén készített diplomaterv \emph{,,Tekintetkövető rendszer fejlesztése képfeldolgozási alapokon''} témában. 

\bigskip

Diplomatervemben egy költséghatékony tekintetkövető rendszer kifejlesztését tűztem ki célul képfeldolgozási módszerekre alapozva, ugyanis számos felhasználási területen bizonyulhat hasznosnak a tekintet követése. Első lépésben kutatási és gyakorlati területen vizsgáltam a rendszer felhasználhatóságát, annak érdekében, hogy megmutassam, nagy potenciál rejtőzik egy pontos és robusztus követési megoldás létrehozásában.

Ezután megvizsgáltam a Hough-transzformációt, a Viola--Jones objektumdetektort és a blob-alapú követést mint lehetséges megoldásokat, annak érdekében, hogy kiválaszthassam a leginkább megfelelő módszert a pupillakövetés megvalósításához, ami az optikai elvű követés alapja.

Blobok detektálására és analízisére alapozva létrehoztam egy működő tekintetkövető rendszert, amelyben képfeldolgozási célokra az OpenCV gépi látás könyvtárat, a felhasználói felület kialakításához pedig a Qt keretrendszert használtam fel.


\newpage

%----------------------------------------------------------------------------
% Abstract in english
%----------------------------------------------------------------------------
\chapter*{Abstract}

The current paper is a thesis from Department of Control Engineering and Information Technology at Budapest University of Technology and Economics in the topic of \emph{,,Development of a Gaze Tracking System Based on Image Processing''}. 

\bigskip

In my thesis research has been done on how to create an inexpensive gaze tracking system using image processing methods. There are several fields of application where a gaze tracking system can be used. First I studied the two main fields --- research and practical --- to show the potential of an efficient and robust tracking solution.

Then I evaluated the Hough-transform, the Viola--Jones cascade classifier and blob-based methods. My objective was to find out which one is the best fit for solving the problem of pupil tracking --- the cornerstone of this area.

Based on blob analysis I've presented a solution for a working gaze tracking system using the OpenCV library for image processing, and the Qt framework for building the application's user interface. 

\vfill
%\clearpage~

\setcounter{page}{1}
\pagenumbering{arabic}
%----------------------------------------------------------------------------
\chapter*{Bevezető}
%----------------------------------------------------------------------------

\begin{verse}
\begin{flushright}
\emph{Hová merűlt el szép szemed világa? \\
Mi az, mit kétes távolban keres?} \\
\dots
\end{flushright}
\end{verse}

Tekintetkövető rendszer fejlesztése összetett feladat. A probléma mind magas, mind alacsony absztrakciós szintről megközelítve megfelelő szakmai felkészültséget követel meg: a rendszer megtervezéséhez, összeállításához és implementációjához olyan önálló mérnöki munka szükséges, ami a ideálissá tette számomra a feladatot diplomatervem témájaként.

\bigskip

A körülöttünk lévő világ megismerésében nyilvánvalóan alapvető szerepet tölt be a látás, ezen belül is a figyelmünk célzott irányítása. A vizuális feldolgozás meglétét és működését a legtöbbször olyannyira készpénznek vehetjük, hogy hajlamosak lehetünk elsiklani a tény fölött, hogy a megértés egyik első lépcsőjeként jelen lévő tekintet követése és analízise milyen mély jelentést hordozhat, legyen szó akár pszichológiai, akár fiziológiai vonatkozásokról.

A tekintet információgazdagságának köszönhetően reményeim szerint a rendszer -- valamint a fejlesztése során megszerzett tapasztalat -- remélhetőleg sok hasznos és érdekes kutatási vagy ipari felhasználásban bizonyulhat alkalmazhatónak.

\bigskip

Diplomatervem \sectref{felhasznalas} fejezetében a tekintetkövető rendszer lehetséges felhasználási területeiről nyújtok rövid áttekintést, a \sectref{tekintetkovetes} fejezetben pedig magát a tekintetkövetést veszem górcső alá a vonatkozó szakirodalomban fellelhető módszerek és a követendő szemmozgások összefoglalásával. Dolgozatom \sectref{elmeleti_alapok} fejezetében a tekintetkövetéshez felhasznált tudományos módszerek elméleti alapjait tárgyalom.

Gyakorlatiasabb vizekre evezve: a \sectref{technologia} fejezet technológiai áttekintésként szolgál, azaz bemutatom benne a fejlesztés során felhasznált szoftver- és hardverkomponenseket. Az \sectref{megvalositas_1} és \sectref{megvalositas_2} fejezetek pedig már kifejezetten a rendszer implementációjáról szólnak. A szétválasztást az indokolta, hogy amíg az \sectref{megvalositas_1} fejezetben magasabb szinten, a felesleges részletek mellőzésével mutatom be a legfontosabb feldolgozási algoritmusokat és igazolom a rendszer használhatóságát, addig a \sectref{megvalositas_2} fejezet első fele kimondottan az alkalmazás architektúrájával és az implementáció részleteivel foglalkozik.

Végezetül a \sectref{megvalositas_2} fejezet második felében dokumentálom az elkészült alkalmazás felhasználói felületét és használati eseteit, valamint összefoglalom és értékelem a munka során gyűjtött tapasztalataimat.
%----------------------------------------------------------------------------
\chapter{Felhasználási lehetőségek}\label{sect:felhasznalas}
%----------------------------------------------------------------------------

A tekintet megfelelő minőségű és robusztus követésének számos gyakorlati felhasználása lehetséges. Elég csak a kutatási területek közül a \emph{perceptuális} (észlelési), vagy \emph{kognitív} (megértési) területekre gondolni, ahol például az olvasás, vagy az alvás folyamatának vizsgálatánál bizonyulhat hasznosnak.

Nem feltétlenül kell azonban a tekintetkövetést laboratóriumok falai közé szorítani: a rendszer számos gyakorlati életből vett felhasználási területen bizonyulhat hasznosnak, például a webergonómia vagy a vezetésbiztonság területein.

\bigskip

Az előző bekezdésekben felvázolt felosztás alapján a fejezet \sectref{tudomanyos} szakaszában a kutatási, míg a \sectref{gyakorlati} szakaszban a gyakorlati felhasználási lehetőségekről nyújtok egy vállaltan nem teljes körű, sokkal inkább az érdeklődés felkeltését célzó összefoglalást.

%,,,,,,,,,,,,,,,,,,,,,,,,,,,,,,,,,,,,,,,,,,,,,,,,,,,,,,,,,,,,,,,,,,,,,,,,,,,,
\section{Kutatási felhasználás}\label{sect:tudomanyos}
%,,,,,,,,,,,,,,,,,,,,,,,,,,,,,,,,,,,,,,,,,,,,,,,,,,,,,,,,,,,,,,,,,,,,,,,,,,,,

%............................................................................
\subsection{Kognitív pszichológia}\label{sect:kognitiv}
%............................................................................

A \textbf{kognitív} (megértési) pszichológia az egyik olyan kutatási terület, amelyben a tekintetkövetés igencsak hasznos eszköz lehet, hiszen a tudományág azt vizsgálja, hogy az ember hogyan látja a világot maga körül, és milyen módon képezi le azt.

\bigskip

Csak egy a lehetséges számtalan megértési probléma közül az \textbf{olvasási folyamat} működésének analízise. Az írott szöveg megértése összetett feladat, ennek a képességnek a hiánya egy tanulási részképességzavar, a \emph{diszlexia} meglétét jelentheti. A diszlexiás állapot felismerésében, a diagnózis megerősítésében, vagy akár az okok felkutatásában lehet hasznos eszköz a tekintetkövető rendszer.

\begin{figure}[!ht]
\centering
\includegraphics[width=100mm, keepaspectratio]{figures/read_saccade.png}
\caption{Szakkádok olvasás közben.\\Forrás: \url{http://bit.ly/C1ia}}
\label{fig:read_saccade}
\end{figure}

Lehetővé válhat a tekintet szakkadikus (egyszerűsítve: a tekintet trajektóriája, részletesen lásd \sectref{szakkadok} szakasz) mozgásának kellően gyors és pontos rögzítése, amely segítségével következtethetünk a megértési folyamat működésére, vagy éppen a működés hibáira. 

\bigskip

Alvásvizsgálat terén leginkább -- nevéből adódóan -- a REM (Rapid Eye Movement) fázis kötődik a szemmozgás követéséhez, ebben az alkalmazásban azonban értelemszerűen optikai elvű követés nem jöhet szóba.

%............................................................................
\subsection{Érzelemdetektálás}\label{sect:erzelem}
%............................................................................

A pupillaátmérő nem csak a fénymennyiség-változás hatására módosulhat. Érzelmi, izgalmi állapotok is előidézőik a változást, mint például félelem, idegesség vagy öröm \cite{altpszicho}. Ez a jelenség szintén potenciális felhasználási lehetőségeket rejt magában. Mivel a pupillareflex akaratlagosan nem koordinálható, állandó fénymennyiség mellett a pupillaméret változásának figyelésével detektálhatóvá válhatnak a fent említett érzelmi állapotok. Ehhez a változás mértékének és sebességének pontos mérése szükséges, ami azonban kellően nagy sebességű kamerával és elfogadható számítási teljesítményt nyújtó hardverrel kielégítő minőségben megtehető lehet. Az alkalmazás ráadásul nem feltétlenül igényli a szem közvetlen közelről (például fejre erősített kamerával) történő felvételét. Megfelelően nagy felbontású forrás esetén az arc-, majd szemrégió automatikus szegmentálása után a felismert zónát felhasználva, akár távolról is történhet a pupillareflex vizsgálata.

%............................................................................
\subsection{Orvosi felhasználás}\label{sect:orvosi_felh}
%............................................................................

Egyes betegségek is okozhatják a pupilla rendellenes méretét vagy viselkedését. Például a ,,miosis'', azaz a szem összehúzódása nemcsak a fent említett okokra vezethető vissza. Rendellenes összehúzódás alakulhat ki bizonyos patológiai állapotok, gyógyszerek, vagy mérgek hatására, sõt a mikrohullámú sugárzásnak kitett szervezet is produkálja ezt a tünetet. A ,,mydraisis'' (a pupilla tágulása) során ugyancsak nem megszokott viselkedés alakulhat ki bizonyos gyógyszerek vagy kábítószerek használatakor, de akár komoly fizikai trauma hatására is a normálisnál jelentősebb mértékű vagy időtartamú lehet a pupilla tágulata. A két szem eltérő méretű pupillája (az ,,anisocoria'') olyan betegségek meglétét jelezheti, mint a Horner-, vagy az Adie-szindróma \cite{altpszicho}. Orvosi szempontból is van tehát mit vizsgálni: a pupilla követésével egyes betegségek, állapotok felismerése, vagy alakulásuk megfigyelése laikus és orvos számára is automatizálható, megkönnyíthető lehet.

\bigskip

Továbbra is orvosi területen maradva a szemmozgás követése és regisztrálása \emph{ontoneurológiai} vizsgálatokban is szerepet kaphat. Az ilyen vizsgálat célja az egyensúlyszerv működésének megfigyelése és értékelése. Az összetett vizsgálat egyes fázisaiban a szemmozgások követése fontos információt hordoz az alany állapotáról, ugyanis a szemmozgató és az egyensúlyi információkat szállító idegpályák szoros kapcsolatban állnak egymással.

%,,,,,,,,,,,,,,,,,,,,,,,,,,,,,,,,,,,,,,,,,,,,,,,,,,,,,,,,,,,,,,,,,,,,,,,,,,,,
\section{Gyakorlati felhasználás}\label{sect:gyakorlati}
%,,,,,,,,,,,,,,,,,,,,,,,,,,,,,,,,,,,,,,,,,,,,,,,,,,,,,,,,,,,,,,,,,,,,,,,,,,,,

%............................................................................
\subsection{Webergonómia}\label{sect:webergonomia}
%............................................................................

Az ergonómia tudománya az ember és technika kapcsolatával, ezen belül is leginkább az említett kapcsolat zökkenőmentesítésével foglalkozik. 

Az ergonómia interdiszciplináris tudomány: egy ága a kapcsolat fizikai valójával foglalkozik (pl. kényelem, akadálymentes használat), egy másik terület inkább a pszichológiával rokonítható módon a fizikailag nem létező (virtuális) felhasználói felületek analízisével foglalkozik. Ugyancsak a grafikus felületek -- azon belül kifejezetten a weboldalak -- analízisével foglalkozik a manapság igencsak felkapott \emph{webergonómia} területe.

\begin{figure}[!ht]
\centering
\includegraphics[width=100mm, keepaspectratio]{figures/google_heatmap.png}
\caption{A Google találati oldalának hőtérképes elemzése.\\Forrás: \url{http://on.mash.to/tHo2ap}}
\label{fig:google_heatmap}
\end{figure}

A megfelelő webes tervezés a \emph{felhasználói élmény} (user experience -- UX) szempontjából döntő fontosságú. Az elhelyezett tartalmak (legyen szó egy egyszerű honlapról, vagy egy összetett webalkalmazásról) közti navigációt úgy kell mind grafikailag, mind strukturálisan megtervezni, hogy a felhasználó intuitívan tudja használni a felületet.

Kellően pontos tekintetkövetéssel vizsgálható lehet, hogy a tervezés során mennyire sikerült a felhasználók igényeinek megfelelő felületet alkotni: könnyen eligazodnak-e rajta, esetleg idejük nagy részét a hibás tervezési döntések következtében kaotikus bolyongással töltik.

A felhasználói aktivitás rögzítésének felhasználására jó példa a \figref{google_heatmap} ábrán látható hőtérképes megjelenítés, amely a Google találati oldalának elemzésével támpontot nyújthat a keresőoptimalizálással foglalkozó szakemberek számára.

%............................................................................
\subsection{Vezetésbiztonság}\label{sect:orvosi_felhasznalas}
%............................................................................

Bár folynak már kutatások és tesztek vezető nélküli, automatizált járművek közúti használatára\footnote{például \emph{Google Driverless Car}, lásd \url{http://bit.ly/PehN2r}}, a humán faktor valószínűleg még nagyon sokáig nem lesz teljesen megkerülhető, ha autóvezetésről van szó.

Vezetésbiztonsági alkalmazásban is elképzelhető lehet a tekintetkövetés alkalmazása. A vezető fókuszának vizsgálatán túl, a követés során mintegy járulékos információként mérhetjük a pislogások gyakoriságát és hosszát is, ezzel felismerhetővé válhat a gépjárművezetés közben lankadó figyelem, és jelezhető, ha fennáll az elalvás veszélye. Az eljárás így hasznosnak bizonyulhat már meglévő elalvásdetektálási módszerek \cite{sleepdet} kiegészítéseként, tovább javítva azok megbízhatóságát.

\begin{figure}[!ht]
\centering
\includegraphics[width=140mm, keepaspectratio]{figures/driving.png}
\caption{\textbf{Bal:} A \emph{Daimler} tekintetkövető kamerája. \\ Forrás: \url{http://bit.ly/14jX6I2} \\
\textbf{Jobb:} A \emph{Seeing Machines} DSS fantázianevű megoldása. \\ Forrás: \url{http://bit.ly/16HLaDE}}
\label{fig:driving}
\end{figure}

%,,,,,,,,,,,,,,,,,,,,,,,,,,,,,,,,,,,,,,,,,,,,,,,,,,,,,,,,,,,,,,,,,,,,,,,,,,,,
\section{Összefoglalás}\label{sect:felh_osszefoglalas}
%,,,,,,,,,,,,,,,,,,,,,,,,,,,,,,,,,,,,,,,,,,,,,,,,,,,,,,,,,,,,,,,,,,,,,,,,,,,,

A fejezetben bemutatott kutatási és gyakorlati területek, problémák felsorolásával éppen csak a felszínét érintettem azon területek, részproblémák halmazának, amelyek körében a tekintetkövetés potenciálisan felhasználható lehet.

Igaz, hogy a dolgozatom írásakor is már több, különféle elveken (összefoglalásukat lásd a \sectref{tekintetkovetes} fejezetben) működő rendszer megtalálható egyetemi laborokban, illetve piacon. Azonban amint említettem, maga a terület, és az egyes projektek anyagi lehetőségei is annyira szerteágazóak, hogy mindenképpen igény mutatkozik egy főleg képfeldolgozással operáló -- azaz komoly hardverköltségek nélküli -- tekintetkövető rendszer fejlesztésére.
%----------------------------------------------------------------------------
\chapter{A tekintetkövetésről}\label{sect:tekintetkovetes}
%----------------------------------------------------------------------------

A tekintetkövető rendszer fejlesztésének megkezdése előtt számos elméleti és gyakorlati szempont veendő figyelembe, hogy a rendszer által nyújtott kritériumok a lehető legjobban közelítsék az elvártat. Nem kerülhető meg az általános technikák, vagy már meglévő megoldások áttekintése a szakirodalomból, aminek segítségével képet alkothatunk a kutatási téma jelenlegi állásáról, és a gyakorlati szempontok figyelembe vételével dönthetjük el, hogy mely módszer mellett tesszük le végül a voksunkat.

\bigskip

A fejezet \sectref{modszerek} szakaszában mindenekelőtt a jelenleg elérhető tekintetkövetési módszereket veszem górcső alá, majd hasonlítom össze őket, hogy kellően megalapozott döntést hozhassak a később felhasználni kívánt technikával kapcsolatban. 

%,,,,,,,,,,,,,,,,,,,,,,,,,,,,,,,,,,,,,,,,,,,,,,,,,,,,,,,,,,,,,,,,,,,,,,,,,,,,
\section{Tekintetkövetési módszerek}\label{sect:modszerek}
%,,,,,,,,,,,,,,,,,,,,,,,,,,,,,,,,,,,,,,,,,,,,,,,,,,,,,,,,,,,,,,,,,,,,,,,,,,,,

Bár már lassan másfél évtizedes, a témában mégis hiánypótló munka Arne John Glenstroupnak és Theo Engell-Nielsennek, a dán Københavns Universitet (Koppenhágai Egyetem) hallgatóinak diplomamunkája \cite{eye_media}. Dolgozatuk jelen szempontból fontos második fejezetét a modern tekintetkövetési technikák bemutatásának és összehasonlításának szentelik, meglátásaik pedig mind a mai napig helytállóak.

\bigskip

Ebben a szakaszban főleg az ő munkájuk alapján szeretném összefoglalni a tekintet követésére felhasználható technikákat, néhol a lényeg kiemelésével, ahol pedig szükséges, a hivatkozott cikk keletkezése óta az idő múlásával érvénytelenné vált adatok, paraméterek aktualizálásával.

%............................................................................
\subsection{Az ideális tekintetkövető}\label{sect:idealis}
%............................................................................

Manapság számos módszer kínálkozik a tekintet követésére. De mégis milyen követelményeket kell teljesíteni az ,,ideális'' tekintetkövetőnek? Scott és Findlay 1993-as munkájukban \cite{scott} Hallett eredményeit \cite{hallett} figyelembe véve meghatározták az ideális tekintetkövető eszköz (vagy rendszer) paramétereit és tulajdonságait.

Ezek szerint az ideális tekintetkövető eszköznek a következő 12 pont által támasztott követelményeket kell teljesítenie:

\begin{enumerate}[a.]
 \item az arc és a fejrégió könnyen hozzáférhető maradjon
 \item ne legyen fizikailag kapcsolatban a vizsgált személlyel
 \item ha szükséges, képes legyen stabilizálni a kapott eredményeket
 \item a rendszer \emph{pontossága} néhány százalékos (1--2 szögperces) eltérést engedjen meg
 \item támogasson legalább egy szögperces \emph{felbontást} másodpercenként, hogy a szempozíció legkisebb változása is követhető legyen; a felbontásnak csak az érzékelő eszköz zaja szabjon határt
 \item támogasson kellően széles \emph{dinamika-tartományt} a szem mozogásainak leképezéséhez
 \item a rendszer időbeli dinamikája legyen megfelelő (jó erősítés, kis fázistolás)
 \item nyújtson \emph{valósidejű} válaszidőt
 \item legyen \emph{invariáns} mindhárom forgási és eltolási szabadságfokra
 \item egyszerűen kiterjeszthető legyen mindkét szem feldolgozására (\emph{binokuláris} vizsgálat)
 \item \emph{kompatibilis} meglévő legyen fej- és testfelvételek használatával
 \item tesztalanyok széles skáláján (pl. nem, kor, rassz szerint, vagy szemüvegesek és szemüveg nélküliek körében) használható legyen
\end{enumerate}

A fent felsorolt követelmények valóban az ideális esetet testesítik meg. Sokszor egy követelmény enyhítésével, vagy figyelmen kívül hagyásával más követelmények kielégítése jelentősen egyszerűsödik. Ha például a \emph{b.} pont szerinti fizikai kontaktust mégis megengedjük, rögzíthetjük a követésre használt kamerát a vizsgálni kívánt alany fejéhez (természetesen kellően diszkrét és kényelmes módon). Ebben az esetben a \emph{i.} pont szerinti invariancia biztosítása máris egyszerűbb feladat, mint külső nézőpontból geometriai transzformációk segítségével végezni ugyanezt.

Több más, látszólag egymásnak ellentmondó követelményt is észrevehetünk az ideális tekintetkövető eszköz 12 pontja között. Komoly megoldandó mérnöki feladatot jelent az ütközések feloldásával az ideális eszközt különböző, valós életben használható megoldásokká alakítani.

\bigskip

A gyakorlatban használható tekintetkövetési megoldások egy lehetséges csoportosítása az alábbi elveken működő iránymeghatározásokat tartalmazza:

\begin{enumerate}
 \item a szem felületének (vagy a felület speciális megvilágításának) alapján, optikai elven 
 \item a szem körüli bőrfelület elektromos potenciáljának mérésével
 \item speciális kontaktlencse használatával
\end{enumerate}

Már csekély megfontolással látszik, hogy minden csoportnak vannak előnyei és hátrányai. Például az \emph{1.} csoportba tartozó megoldások igénylik a legkevésbé -- többnyire egyáltalán nem -- a fizikai kontaktust a vizsgált személlyel. Azonban elképzelhető, hogy ennek az az ára, hogy az ebbe a csoportba tartozó megoldások nem veszik fel a versenyt a másik két alapelv szerint fejlesztett rendszerekkel pl. a pontosság, vagy valamely más kritérium tekintetében.

Ennek megfelelően szükség van a lehetőségek számba vételére, értékelésére, majd összehasonlítására, hogy a legmegfelelőbb technikát tudjuk kiválasztani, ha tekintetkövető eszköz (alkalmazás) fejlesztésébe fogunk. A következő szakaszokban ezért röviden ismertetem a fenti kategóriákba eső megoldások alapjait, majd összehasonlítom jellemző tulajdonságaikat, valamint a használatukkal elérhető fontos pontossági- és sebességértékeket.

%............................................................................
\subsection{Optikai elvű követés}\label{sect:optikai}
%............................................................................

Az optikai elvű követés során -- a nevéből adódóan -- optikai úton próbáljuk meghatározni a tekintet irányát. Ez történhet speciális megvilágítás nélkül, pl. az írisz (pontosabban az írisz és az ínhártya közti határvonal, a limbus), vagy a pupilla követésével.

A másik lehetséges megoldásra a szem speciális, többnyire infravörös fénnyel történő megvilágítása, majd a szemgolyó felületén megjelenő tükröződések, visszaverődések vizsgálata kínálkozik. Az infravörös megvilágítás előnye a látható fénnyel szemben egyrészt az, hogy nem zavarja a vizsgált személyt, nem vonja el a figyelmét, másrészt pedig infraszűrők használatával a látható fény tartományába eső változásokkal a megfelelő körülmények között (lehetőleg kevés napfény a magas infratartalma miatt) invariánssá tehető.

%. . . . . . . . . . . . . . . . . . . . . . . . . . . . . . . . . . . . . .
\subsubsection{Limbuskövetés}\label{sect:limbusz}
%. . . . . . . . . . . . . . . . . . . . . . . . . . . . . . . . . . . . . .

Az írisz és az ínhártya közötti határvonal a többnyire nagy intenzitásváltozás miatt könnyen detektálható lehet. Ugyanakkor meg kell jegyeznünk, hogy az esetek jelentős részében a limbus számottevő része lehet a szemhéjak takarásában. Ennek megfelelően a technika csak vízszintes helyzet és mozgás követésére alkalmas kielégítő pontossággal \cite{scott}.

Klasszikus esetben ez a követési technika a limbus \emph{relatív} fejhez képest helyzetén alapul, ezért a vizsgálat közben a fejmozgások teljes hanyagolását, esetleg a tekintetkövető eszköz fejhez rögzítettségét igényli.

%. . . . . . . . . . . . . . . . . . . . . . . . . . . . . . . . . . . . . .
\subsubsection{Pupillakövetés}\label{sect:pupilla}
%. . . . . . . . . . . . . . . . . . . . . . . . . . . . . . . . . . . . . .

A pupillakövetési technika hasonló az előbb említett limbuskövetéshez, de némi többlet előnyt is magában hordoz. Az egyik ilyen előny, hogy a pupilla közel sincs olyan nagy részben a szemhéjak takarásában, mint a limbus, ennek következtében függőleges irányú követés is megvalósítható lehet. A másik előny, hogy az írisz és a pupilla közti határvonal jóval élesebb, mint az írisz-ínhártya határa, ezért jobb felbontással, pontosabban tudjuk meghatározni a nézeti irányt.

Az előző technikához képest azonban meg kell említeni a felmerülő nehézségeket is. A kontraszt ugyan lehet, hogy nagyobb az írisz-pupilla határon, azonban az intenzitáskülönbség nem olyan jelentős, mint a limbus környezetében. Az sem elhanyagolható szempont, hogy a pupilla átmérője is szükségszerűen kisebb, mint az íriszé, vagyis a forrásnak relatívan nagyobb felbontásúnak kell lennie, hogy azonos pixelnyi méretű pupillát és íriszt detektálhassunk; ellenkező esetben a kisebb abszolút méret a pontosság rovására mehet.

%. . . . . . . . . . . . . . . . . . . . . . . . . . . . . . . . . . . . . .
\subsubsection{Visszaverődés-alapú követés}\label{sect:visszaverodes}
%. . . . . . . . . . . . . . . . . . . . . . . . . . . . . . . . . . . . . .

A szemet (infravörös) fénnyel megvilágítva több tükröződés is megfigyelhető lesz a szemlencse és a szaruhártya határán: ezek az úgynevezett \emph{Purkinje-képek} (lásd \figref{purkinje} ábra).

\begin{figure}[!ht]
\centering
\includegraphics[width=110mm, keepaspectratio]{figures/purkinje_kepek.png}
\caption{A \emph{Purkinje-képek} elhelyezkedése. Forrás: \url{http://www.fourward.com/dconcept.htm}}
\label{fig:purkinje}
\end{figure}

Ezen visszavert képek intenzitása sorrendben egyre csökken, az első azonban (közkeletű nevén az úgynevezett \emph{,,csillanás''}, angolul \emph{glint}) még viszonylag egyszerűen detektálható. Infravörös megvilágításban ugyancsak egyszerű a megfelelő kamerával a pupilláról visszavert fény detektálása -- a pupilla a környezeténél jóval nagyobb mértékben veri vissza az infravörös fényt, az infraképen egy ,,fényes'', kontrasztos objektumot alkotva.

A fenti két objektum egymáshoz viszonyított \emph{relatív} helyzetéből következtethetünk a tekintet irányára, ugyanis az első Purkinje-kép világos pontja, illetve a pupilla kontrasztos ellipszise egymással összefüggésben mozdulnak el a fej vagy a szem mozgatásának következtében.

Az előző bekezdésben foglaltakból látszik, hogy a technika nem igényeli a fej mozdulatlanságát, vagy a képfelvevő eszköz fejhez rögzítését. Ezen előnye azonban egy megkötést is magával hordoz: egyszerű algoritmusok segítségével kb. $\pm12$--$15$ foknyi szabadsága van a felhasználónak a fejmozgásokra \cite{scott}, nagyobb mértékű mozgások esetén ugyanis komplexebb matematikai számítások szükségesek a követett objektumok mozgásának modellezéséhez.

\bigskip

A visszaverődés-alapú megoldásokkal rokonítható még az a módszer, amikor szintén az első Purkinje-kép erős visszaverődését detektálva a képen, kivágunk egy viszonylag kicsi (párszor tíz pixel nagyságrendű) részt, a csillanással a középpontban. Az így kapott képről egy neurális hálózat dönti el, hogy milyen nézeti irányhoz tartozik \cite{baluja}.

A neurális hálózatot természetesen be kell tanítani a használhatóság érdekében. A betanítási procedúra első lépéseként tanítóképeket kell generálni, általános esetben pár percet vesz igénybe, ami alatt a követnie kell egy, a képernyőn megjelenő jelzést. A tanítóképek birtokában ezután a hálózat betanítható, ez a jelenlegi technikai szint mellett kb. tíz perces nagyságrendű időt vesz igénybe. Azonban azonos körülmények és tesztszemély esetén a tanítást nem kell máskor újra elvégezni.

A neurális hálózat előnye a betanítás után, használat közben mutatkozik meg leginkább: felépítéséből adódóan a háló nagyon rövid idő alatt ,,döntést tud hozni'', így a valósidejű feldolgozási sebesség kritériuma mindenképpen kielégített lesz. A döntés eredménye ráadásul akár közvetlenül felhasználható: mindössze úgy kell megterveznünk a rendszert, hogy a neurális háló kimenet közvetlenül egy kétdimenziós koordinátát szolgáltasson a felhasznált képernyő koordináta-rendszerében.

A módszer egy másik előnye, hogy nem igényel közeli, nagy felbontású képet a szemrégióról, egy átlagos felbontású kamerával, pl. kartávolságból is elegendő nagyságú lesz a szemrégió. Ennek egyik folyományaként -- hogy nagyobb látószögű képek használhatók -- adódik, hogy viszonylag nagy mozgási szabadsága lehetséges a vizsgált személy fejének tekintetében, anélkül, hogy a kamerát újra kéne pozicionálnunk.

Ennek a szabadságnak azonban ára van: ha a kalibrálási fázis során több fejpozícióból is rögzítettünk tanító képeket, akkor a betanítás után a neurális hálózat jó eséllyel fogja felismerni különböző fejpozíciókban is a tekintet irányát; a mozgási szabadság azonban a pontosság csökkenésével jár. A pontosság egyébként is érzékeny pontja az eljárásnak, ezt pedig éppen úgy növelhetjük, ha tanítás és előhívás közben \emph{nem} engedjük meg, hogy a fejpozíció változzon.

%. . . . . . . . . . . . . . . . . . . . . . . . . . . . . . . . . . . . . .
\subsubsection{Purkinje-képek követése}\label{sect:purkinje}
%. . . . . . . . . . . . . . . . . . . . . . . . . . . . . . . . . . . . . .

Egy további lehetséges optikai elven működő tekintetkövetési technikát mutatott be Müller \emph{et al} 1993-ban \cite{muller}, \emph{,,Kettős Purkinje-kép''} (Dual-Purkinje Image) módszer néven. Az eljárás lényege, hogy a már említett első és negyedik Purkinje-kép egymáshoz viszonyított helyzetéből számítja a tekintet irányát. Megfelelő matematikai modell esetén a módszer rendkívül pontos, azonban a negyedik Purkinje-kép alacsony intenzitása miatt hatványozottan érzékeny a megvilágítási problémákra.

%............................................................................
\subsection{Elektromos potenciál-alapú követés}\label{sect:potencial}
%............................................................................

Az eddigiektől egy merőben eltérő megközelítése a problémának az \emph{elektro-okulográfia} (EOG). Az eljárásból nyerhető elektro-okulogram rögzítése pl. megismerési és kognitív folyamatok, a vizuális információfeldolgozás, vagy az alvásvizsgálat terén szokásos.

\begin{figure}[!ht]
\centering
\includegraphics[width=80mm, keepaspectratio]{figures/eog.png}
\caption{Az EOG eljárás elektródái. Forrás: \url{http://www.metrovision.fr/mv-po-notice-us.html}}
\label{fig:eog}
\end{figure}

A módszer a szemgolyó elülső és hátulsó pólusa közötti potenciálkülönbség mérésén alapul, amellyel mind függőleges, mind vízszintes irányban követhető a szem mozgása, de csak körülbelül $1$--$2$ fok pontossággal. A használata azonban korántsem mondható egyszerűnek: az potenciálváltozást érzékelő elektródák (lásd \figref{eog}) miatt szükséges fizikai kontaktuson kívül a rendszer kalibrációja rendkívül hosszadalmas, és hozzáértést igénylő feladat.

%............................................................................
\subsection{Követés speciális kontaktlencse használatával}\label{sect:kontakt}
%............................................................................

A szem helyzetének, ebből közvetve a tekintet irányának számításához speciális kontaktlencséket is használhatunk. Az egyik lehetséges megoldás, hogy a lencse anyagában olyan véseteket alakítanak ki (tipikusan valamilyen könnyen felismerhető mintázatot), amelyek a fénytörés felhasználásával megkönnyítik a szem helyzetének meghatározását.

Ha azonban egy megfelelően kis méretű indukciós tekercset is sikerült a kontaktlencse anyagába ágyazni, akkor a fej körül generált nagyfrekvenciás elektromágneses mezőkkel az tekercs helyzete közvetlenül is könnyen meghatározható. Az eljárás körülményessége azonban kétségessé teszi a technika laboratóriumokon kívüli felhasználását.


%............................................................................
\subsection{A tekintetkövetési módszerek összehasonlítása}\label{sect:tekintet_osszehas}
%............................................................................

A \tabref{osszehas} táblázat tartalmazza az ebben a szakaszban felsorolt tekintetkövetési módszerek összehasonlítását. Az összehasonlítás szempontjai a módszer által igényelt kontaktust, az általa nyújtott pontosságot, illetve felbontást (hogy a bemeneti képen mekkora minimális elmozdulás jelent változást a kimeneti pozícióban) jelentik. 

\begin{table}[ht]
	\centering
	\caption{A felsorolt tekintetkövetési módszerek összehasonlítása.} \label{tab:osszehas}
	\begin{tabular}{ l || c | c | c | c | c | c | c }
	 & kontaktus & pontosság & felbontás \\ \hline \hline
	limbuskövetés & pl. áll-tartó & $1$--$7^\circ$ & $0,\!1^\circ$ \\
	pupillakövetés & nincs & $0,\!003^\circ$ & $0,\!005^\circ$ \\
	visszaverődés alapú & nincs & $0,\!5$--$2^\circ$ & jó \\
	neurális hálózat & nincs & $1,\!5\circ$ & -- \\
	kettős Purkinje-képek & nincs & $0,\!017^\circ$ & $0,\!25^\circ$ \\ 
	elektro-okulográfia & elektródák & $\pm1,\!5$--$2^\circ$ & jó \\
	kontaktlencse & kontaktlencse & $0,\!08^\circ$ & $0,\!017^\circ$ \\
	\end{tabular}
\end{table}
%----------------------------------------------------------------------------
\chapter{Elméleti alapok}\label{sect:elmeleti_alapok}
%----------------------------------------------------------------------------

\texttt{+++ bevezeto a fejezethez +++}

\texttt{+++ Hough-t megvagni? +++}

\bigskip

\texttt{+++ melyik szakaszban mi van +++}

%,,,,,,,,,,,,,,,,,,,,,,,,,,,,,,,,,,,,,,,,,,,,,,,,,,,,,,,,,,,,,,,,,,,,,,,,,,,,
\section{A Hough-transzformáció}\label{sect:hough}
%,,,,,,,,,,,,,,,,,,,,,,,,,,,,,,,,,,,,,,,,,,,,,,,,,,,,,,,,,,,,,,,,,,,,,,,,,,,,

Képfeldolgozási szûrõk, függvények segítségével viszonylag könnyen elõállíthatjuk a képtérben számunkra érdekes pontok halmazát. Azokat a pontokat, amelyek valamilyen magasabb absztrakciós szintû képjellemzõhöz kapcsolódnak: például a tárgyak, alakzatok formáját megadó kontúrok pontjait. Ha viszont szeretnénk ,,megérteni'' is a képet, az ilyen formák automatikus és robusztus felismerése szinte elengedhetetlen.

A valóságban a kontúrok pontjai azonban csak többé-kevésbé illeszkednek ideális formákra (egyenesekre, körökre, vagy egyéb görbékre): az éleket torzíthatja zaj, egyes élpontok hiányozhatnak, vagy a felismerni kívánt formák kismértékben el is térhetnek az ideálistól. A kép analízise szempontjából azonban ezzel együtt is nagyon értékes információt rejthetnek magukban, ezt az információt pedig hasznos lenne kinyerni. Viszont egyáltalán nem triviális probléma a valamilyen szempontból összetartozó (például egy egyenesre, vagy egy körvonalra esõ) pontok csoportosítása.

\begin{figure}[!ht]
\centering
\includegraphics[width=67mm, keepaspectratio]{figures/houghline_building_1.png}\hspace{1cm}
\includegraphics[width=67mm, keepaspectratio]{figures/houghline_building_2.png}
\caption{Egyenesek keresése Hough-transzformáció használatával.\\Forrás: \url{http://href.hu/x/c91h}}
\label{fig:houghlines}
\end{figure}

A Hough-transzformáció erre a problémára kínál megoldást. Feladata egyszerû formák, úgymint egyenesek, körök, ellipszisek keresése képeken. Az alakzatok keresését a transzformáció egy ún. paramétertérben végzi egy szavazási mechanizmus segítségével. A felismert potenciális objektumok az akkumulátortér -- a paramétertér egyfajta futás közbeni leképezése -- lokális maximumai alapján adódnak.

\bigskip

Történelmi áttekintésként megjegyezhetjük, hogy a Hough-transzformációt Paul~Hough alkotta meg 1959-ben buborékkamra-fényképek gépi analíziséhez \cite{hough_eredeti}. Ebben a cikkben Hough még csak egyenesek felismerésérõl beszél. A manapság használt transzformációt Richard~Duda és Peter~Hart fejlesztette ki 1972-ben ,,általánosított Hough-transzformáció'' néven \cite{hough_duda}, tíz évvel azután, hogy Paul Hough szabadalmaztatta módszerét. Az õ kiegészítésük már lehetõvé tette, hogy a transzformáció nemcsak egyenesek, hanem körök és más analitikusan leírható görbék keresésére is felhasználható legyen.

A gépi látás felhasználói körében azonban igazán csak Dana~H.~Ballard 1981-es cikke \cite{hough_ballard} nyomán lett népszerû, aki a transzformáció még további felhasználási lehetõségeit vetette fel. Eredményeinek köszönhetõen a módszer analitikusan nem (vagy csak nehezen) leírható formák keresésére is alkalmazhatóvá vált. Félreértésekre adhat okot, hogy mind Duda és Hart, mind Ballard az ,,általánosított'' megnevezést használja a módszer általuk kitalált kiegészítéseinek. A továbbiakban az ,,általános'' jelzõt a Duda--Hart-féle kiegészítés megnevezésére használom, ahol Ballard módszerérõl esik szó, azt külön jelzem.

\bigskip

A transzformáció általános formájában tehát analitikusan leírható formák keresésére használható. A legegyszerûbb ilyen forma az egyenes, ennek segítségével érthetõ meg legkönnyebben az algoritmus mûködési elve. A \sectref{egyenesek_keresese} szakaszban ezért az egyenesek keresésének elméletét foglalom össze. Ezen elv késõbb már könnyen kiterjeszthetõ tetszõleges formák keresésére, különös tekintettel a pupilla-keresés szempontjából fontos körkeresési problémára, amelyre a \sectref{korkereses} szakaszban térek ki. A \sectref{kiegeszitesek} szakaszban szót ejtek még a módszer további kiegészítési lehetõségeirõl, majd a \sectref{hough_osszefoglalas} szakaszban összefoglalom a Hough-transzformáció használatának korlátait, elõnyeit és hátrányait.


%............................................................................
\subsection{Egyenesek keresése}\label{sect:egyenesek_keresese}
%............................................................................

%. . . . . . . . . . . . . . . . . . . . . . . . . . . . . . . . . . . . . .
\subsubsection{Egyenes-reprezentációk}\label{sect:egyenes_reprezentaciok}
%. . . . . . . . . . . . . . . . . . . . . . . . . . . . . . . . . . . . . .

Egyenesek leírására többféle modell használható. Az egyenes ,,klasszikus'' modellje Descartes-koordinátarendszerben az

\begin{align}\label{eq:egyenes_klasszikus}
y = m \cdot x + b
\end{align}

reprezentáció, ahol $ m $ az egyenes meredeksége (iránytangense), a $ b $ konstans az ordinátatengely-metszet, vagyis az egyenes és az $ y $ tengely metszéspontja.

A \eqref{egyenes_klasszikus} egyenlettel megadott modellel az a probléma, hogy a koordinátarendszer $ y $ tengelyével párhuzamos, vagy ahhoz közelítõ egyenesek leírására nem használható $ m $ végtelen nagy értéke miatt. Nem szerencsés azonban az ilyen ,,függõleges'' egyenesek kizárása a számításból. Ennek kiküszöbölésére -- jelen esetben -- jobb modellként segítségül hívhatjuk a \textbf{Hesse-féle normálalakos} reprezentációt. Ez az

\begin{align}\label{eq:egyenes_hesse}
r = x \cdot \cos \theta + y \cdot \sin \theta
\end{align}

egyenlettel adja meg a kívánt egyenest, ahol $ r $ az origótól mért távolságot, $ \theta $ pedig az egyenes pozitív valós féltengellyel bezárt szögét jelenti (lásd \figref{repr_line} ábra). Ezzel a megvalósítással már nem ütközik akadályba az $ y $ tengelyhez közelítõ egyenesek modellezése.

\begin{figure}[!ht]
\centering
\includegraphics[width=80mm, keepaspectratio]{figures/repr_line.png}
\caption{Egyenes jellemzése $ r $ és $ \theta $ paraméterekkel.}
\label{fig:repr_line}
\end{figure}

\bigskip

Érdekességként megemlíthetõ, hogy bár nyilvánvalónak látszik a Hesse-féle normálalak elõnyösebb volta, Paul Hough eredeti szabadalmában\footnote{U.S. Patent 3,069,654 -- \url{http://www.google.com/patents?q=3069654}} mégis az $ y = m \cdot x + b $ formát használta az egyenesek jellemzésére. Az $ (r, \theta) $ paraméterezés Duda és Hart már hivatkozott cikke nyomán vált általánosan használttá. \cite{hough_duda}

%. . . . . . . . . . . . . . . . . . . . . . . . . . . . . . . . . . . . . .
\subsubsection{A paramétertér és az akkumulátor}\label{sect:parameterter_akkumulator}
%. . . . . . . . . . . . . . . . . . . . . . . . . . . . . . . . . . . . . .

Minden képtérbeli egyenes tehát \eqref{egyenes_hesse} felhasználásával megfeleltethetõ a paramétertérben egy $ (r, \theta) $ párral jellemzett pontnak. Az egyértelmû megfeleltetés végett két lehetõségünk is adódik a paraméterek korlátainak megválasztására. Ezek formálisan

\begin{align}\label{eq:param180}
\theta \in \left[ 0,180^{\circ} \right) \quad \wedge \quad r \in \mathbf{R}
\end{align}

vagy

\begin{align}\label{eq:param360}
\theta \in \left[ 0,360^{\circ} \right) \quad \wedge \quad r \geq 0
\end{align}

Az $ (r, \theta) $ párok terét \textbf{paramétertérnek}, vagy \textbf{Hough-térnek} hívjuk. A paramétertér dimenzióját az ismeretlen paraméterek száma adja, egyenesek esetében ez tehát kettõ.

Egy $ (x,y) $ ponton keresztül végtelen sok egyenes húzható, és minden egyenes kielégíti a \eqref{egyenes_hesse} összefüggést. Ezek az egyenesek a paramétertérben ábrázolva egy szinusz-jellegû görbét alkotnak. Természetesen nem határozhatunk meg tetszõleges pontossággal egy pontot, és nem is vehetünk számításba minden (végtelen számú) olyan egyenest, ami az adott ponton átmegy. Szükség van ezért a paramétertér diszkretizálására.

\bigskip

A paramétertér gyakorlati (kvantált) reprezentációja az \textbf{akkumulátor}. Az akkumulátor egy tömb, dimenziószáma megegyezik a paramétertér dimenziószámával.

A tömbök indexelése egész számokkal történik ezért az $ r $ paramétert egész számokra célszerû kvantálnunk. A $ \theta $ paraméter kvantálásához pedig a végtelen sok egyenes figyelembe vétele helyett az $ (x,y) $ koordinátán keresztül húzhatunk egyeneseket $ \varDelta \theta $ fokonként. Ebbõl következõen a \eqref{param180} vagy \eqref{param360} összefüggések alapján az akkumulátor ezen dimenzióját $ 180^{\circ} / \varDelta \theta $, illetve $ 360^{\circ} / \varDelta \theta $ darab diszkrét részre osztja. A $ \varDelta \theta $ paraméter megválasztásával, a módszer ,,finomságát'' növelhetjük. Például $ \varDelta \theta = 5^{\circ} $ választás esetén a transzformáció csak a pozitív valós féltengellyel $ 0, 5, 10, ... $ fokot bezáró egyenesek felismerésére képes. Túl kis érték választása esetén viszont az eljárás memória- és idõigénye fog az egekbe szökni. A $ \varDelta \theta $ érték gondos megválasztásával tehát a pontosság és gyorsaság egy lehetõleg optimális értékét kell meghatároznunk.

Gyakorlati szabályként elmondható, hogy 1 fok pontosságnál többre ritkán van szükség. Túl vastag vonalak (amik pedig logikailag összefüggõ egyenesek), zajos kép, vagy túl rövid vonalszakaszok esetén így is romlik a felismerés esélye. Hogy miért, arra a következõ szakasz -- a szavazási mechanizmus ismertetése -- ad választ.

%. . . . . . . . . . . . . . . . . . . . . . . . . . . . . . . . . . . . . .
\subsubsection{A keresés folyamata}\label{sect:kereses_folyamata}
%. . . . . . . . . . . . . . . . . . . . . . . . . . . . . . . . . . . . . .

A továbbiakban tételezzük fel, hogy a kép bináris, és megfelelõen elõfeldolgozott (pl. élkeresés, küszöbözés), azon már csak a felismerni kívánt egyenesek pontjai találhatóak. A háttér képpontjait jelölje a 0 érték, az érdekes pontokat tartalmazó elõtér képpontjai legyenek 1 értékûek.

A keresés folyamán bejárjuk a képteret pixelrõl pixelre. A 0 értékû képpontok biztosan nem részei egy képen található egyenesnek sem, ezért ezekkel nem kell foglalkozunk. Ha 1 értékû pixelt találunk az $ (x,y) $ helyen, az potenciálisan része lehet egy, a képen található egyenesnek. Az akkumulátor minden $ \theta $ értékhez (pl. fokonként) meghatározzuk az $ (x,y) $ ponton átmenõ, $ \theta $ irányú egyenes origótól vett $ r $ távolságát a \eqref{egyenes_hesse} összefüggés alapján. Az akkumulátortömb értékét minden így kiszámolt $ (r,\theta) $ indexekkel jelölt helyen megnöveljük ($ r $ meghatározásánál természetesen a tömb kvantáltságát figyelembe véve). Ezt a folyamatot úgy is mondhatjuk, hogy az $ (x,y) $ pont a kiszámított $ (r, \theta) $ pontok halmazára, vagyis ezen paraméterek által a képtérben képviselt egyenesekre \textbf{,,szavaz''}.

Az algoritmus elsõ fázisának lefutása után az akkumulátortömböt kell vizsgálnunk. Optimális esetben a képen egy egyenesre esõ pixelek mindegyike (a többi más szavazatuk mellett) szavazott a ,,valódi'' egyenest jelentõ $ (r, \theta) $ párra. Az akkumulátortömb maximális értékû $ (r, \theta) $ elemei adják meg tehát a képen található egyeneseket. A zaj, az esetleg több pixel széles vonalak, és a kvantálási pontatlanság miatt azonban célravezetõbb \textbf{lokális maximumokat} (lásd \figref{houghparam} ábra) keresni az akkumulátortömbben így meghatározva a \textbf{legvalószínûbb} egyenesek $ r $ és $ \theta $ paramétereit.

\begin{figure}[!ht]
\centering
\includegraphics[width=67mm, keepaspectratio]{figures/houghparam_1.png}\hspace{1cm}
\includegraphics[width=67mm, keepaspectratio]{figures/houghparam_2.png}
\caption{A képtér (balra) és az ebbõl származó paramétertér (jobbra) egyenesek keresése során.\\Forrás: \url{http://href.hu/x/c91i}}
\label{fig:houghparam}
\end{figure}

Az elõzõ szakasz végén felvetett problémát most már megválaszolhatjuk. A túl nagy zaj miatt megnõ azon képpontok száma, amik nem részei egy egyenesnek sem. Azonban szavazni ezek a képpontok is szavaznak, ezzel mintegy ,,háttérzajt'' adva az akkumulátortömbnek. Túl vastag vonalak esetén pedig több, látszólag helyes paraméterezésû egyenes adódna ugyanarra a vonalra (pl. egymással a vonalon belül párhuzamos egyenesek, de kellõen vastag vonal esetén akár ,,átlós'' egyenesek is). Az elsõ esetben az akkumulátortömb elemeinek átlagos értéke nõ meg, a második esetben pedig a sok egymáshoz közeli szavazat miatt lokális maximumok értékei kerülnek közelebb az átlagos tömbértékekhez. Mindkét esetben nehezebbé válik a lokális maximumok pontos meghatározása, ez pedig drasztikusan csökkentheti a keresés pontosságát.

%............................................................................
\subsection{Körkeresés}\label{sect:korkereses}
%............................................................................

%. . . . . . . . . . . . . . . . . . . . . . . . . . . . . . . . . . . . . .
\subsubsection{Körök reprezentációja}\label{sect:korok_reprezentacioja}
%. . . . . . . . . . . . . . . . . . . . . . . . . . . . . . . . . . . . . .

A körökhöz megfelelõ reprezentáció megtalálásához még annyit sem kell törnünk a fejünket, mint az egyeneseknél tettük. Descartes-koordinátarendszerben az $ (x_{0}, y_{0}) $ középpontú $ r $ sugarú körvonal pontjait az

\begin{align}\label{eq:kor_koogeom}
( x - x_{0})^{2} + (y - y_{0})^2 = r^2
\end{align}

egyenlet adja meg (\figref{repr_circle} ábra). Szögfüggvények használatával az \eqref{kor_koogeom} egyenletet átalakíthatjuk

\begin{align}\label{eq:kor_param}
x &= x_{0} + r \cdot \cos \theta \nonumber \\
y &= y_{0} + r \cdot \sin \theta
\end{align}

formára, ahol $ (x_{0}, y_{0}) $ szintén a kör középpontjának koordinátái, $ r $ a sugár, $ \theta $ pedig a körvonal egy pontját a kör középpontjával összekötõ szakasz pozitív valós féltengellyel bezárt szögét jelenti.

\begin{figure}[!ht]
\centering
\includegraphics[width=80mm, keepaspectratio]{figures/repr_circle.png}
\caption{Körök reprezentációja.}
\label{fig:repr_circle}
\end{figure}

%. . . . . . . . . . . . . . . . . . . . . . . . . . . . . . . . . . . . . .
\subsubsection{A paramétertér körök esetében}\label{sect:korok_parameterter}
%. . . . . . . . . . . . . . . . . . . . . . . . . . . . . . . . . . . . . .

Látható, hogy körök esetében már csak akkor elegendõ két paraméter -- az $ (x_{0}, y_{0}) $ középpont-koordináták -- a forma leírásához, ha elõre adott sugarú köröket keresünk, vagyis $ r $ konstans. Ez azonban csak bizonyos esetekben használható, általános esetben túl nagy megkötést jelent. Tehát a paramétertér dimenziószámát növelnünk kell: az $ r $ sugárparaméter dimenziójával együtt az már \textbf{három dimenziós lesz}.

A paraméterteret az akkumulátor tömb mivoltából adódóan itt is egész számokra kell kvantálnunk. Körkeresés esetén az akkumulátor tehát egy három dimenziós tömb, amely a középpont két koordinátájának és a sugárnak egész értékeivel indexelhetõ. Már a pontos algoritmus ismerete nélkül is látszik, hogy a potenciális körök keresése jóval komplexebb mûvelet az ismeretlen paraméterek magasabb száma miatt.

%. . . . . . . . . . . . . . . . . . . . . . . . . . . . . . . . . . . . . .
\subsubsection{A körkeresés folyamata}\label{sect:korok_keresese}
%. . . . . . . . . . . . . . . . . . . . . . . . . . . . . . . . . . . . . .

Az alaphelyzet legyen az egyenesek keresésekor felállított: a felismerni kívánt körvonalak képpontjai legyenek 1, a háttér felismerés szempontjából érdektelen képpontjai pedig legyenek 0 értékûek.

A bejárás során 1 értékû pixelt találva jóval több számítani valónk van, mint egyenesek esetében. Az akkumulátor \textbf{összes lehetséges $ r $ sugárparaméterére} meg kell határoznunk a potenciális körök középpontjait. Ehhez felhasználjuk az \eqref{kor_param} összefüggéseket $ r $ és $ \theta $ ismeretében ($ \theta $ lehetséges értékeinek megválasztása -- tipikusan például fokonként -- azonos elven történhet a \sectref{parameterter_akkumulator} szakaszban olvashatóakkal).

Hogy próbálhatjuk meg csökkenteni a rengeteg számítást? Egyfelõl \textbf{korlátozhatjuk} $ r $ lehetséges értékeinek halmazát. Az értékeknek $ r_{min} $ alsó és $ r_{max} $ felsõ korlátot adva a keresést a két érték közötti sugarú körökre szûkíthetjük. Némi elõzetes információ birtokában (körülbelül, vagy a kép oldalainak arányában mekkora köröket kell keresnünk) jelentõsen csökkenthetõ a számítási idõ- és tárigény.

Másfelõl pedig felhasználhatunk lokális információkat is: a vizsgálat alatt lévõ képpont egy adott ablaknyi környezetét vizsgálva, meghatározhatjuk a körvonal adott pontbeli \textbf{gradiensét}. Ez az információ felhasználható a középpont keresésében, ugyanis a kör középpontjának a körvonalpont normálisán kell feküdnie. Így, ha nem is egyetlen kitüntetett irányra, de a számítási pontatlanságot és a zajt figyelembe véve egy szûk tartományra korlátozhatjuk a keresett körközéppontok irányát. Ez a választott tartomány méretével fordított arányosságban csökkenti a szükséges számítások számát. Példának okáért, ha a kiszámított gradiens segítségével akár csak 90 fok pontossággal meg tudjuk határozni a pontban vett normális irányát, a számítások háromnegyedét nem kell elvégeznünk, ráadásul az akkumulátortömb sem telítõdik feleslegesen olyan paraméterekkel, amelyekbõl biztosan (vagy elég nagy valószínûséggel) nem fog helyes megoldás születni.

%. . . . . . . . . . . . . . . . . . . . . . . . . . . . . . . . . . . . . .
\subsubsection{Variációk körkeresésre}\label{sect:korok_kiegeszites}
%. . . . . . . . . . . . . . . . . . . . . . . . . . . . . . . . . . . . . .

A körkeresésre használt Hough-transzformáció további kiegészítéseivel találkozhatunk H. K. Yuen \textit{et al} hivatkozott cikkében \cite{hough_circles}. A cikkben az elõzõ alfejezetben ismertetett sztenderd változat mellett még további négy módosulat vizsgálatát és ezek összehasonlítását végzik el a szerzõk. Ezek közül a cikkben ,,2-1 Hough-transzformációnak'' néven hivatkozott módszert emelném ki, a késõbbiek során ugyanis ennek fontos szerepe lesz. Elnevezésként a cikkben használt ,,21HT'' rövidítést fogom használni.

A \textbf{21HT} módszer a \cite{hough_21_davies} és \cite{hough_21_illingworth} számon hivatkozott cikkekben volt elõször használatos. A kiegészítés felhasználja a rendelkezésre álló gradiens-információt (lásd: \sectref{korok_keresese}), és ennek ismeretében pedig a problémát két részre osztja. Mivel a kör középpontjának a körvonalpontok normálisán kell feküdnie, ezen normálisok közös metszéspontja valójában tehát meghatározza a középpontot. Egy kétdimenziós akkumulátortömb pedig elég minden pont saját normálisára esõ szavazatainak nyilvántartásához. Ezután a sugár meghatározása a következõ módszerrel történhet: meghatározzuk minden pont és az elõzõ lépésben kiszámított középpont-jelölt távolságát, majd ezekbõl az információkból egy sugár-hisztogramot állítunk elõ. Ennek vizsgálatával már meghatározhatjuk a középpontokhoz tartozó sugarakat is. A módszer tárigénye az eredeti megközelítéshez képest jóval kisebb, hiszen csak egy \textbf{2D akkumulátort} és egy \textbf{1D hisztogramot} kell használnunk -- innen a módszer ,,2-1'' elnevezése.

Eredmények tekintetében az összehasonlító cikk szerint a 21HT módszer felveszi a versenyt a sztenderd megoldással. Igaz, hogy a kétfázisú számítás miatt az elsõ fázisban összeszedett hiba szükségszerûen rárakódik a második fázisra, ezt a pontatlanságot ellensúlyozni tudja a módszer jelentõsen kisebb tárigénye.

%............................................................................
\subsection{Kiegészítések}\label{sect:kiegeszitesek}
%............................................................................

A meglévõ módszer továbbfejlesztésekor két irányba is elindulhatunk:

\begin{enumerate}
 \item egyrészt növelhetjük az algoritmus teljesítõképességét,
 \item másrészt bõvíthetjük a felismerhetõ formák halmazát.
\end{enumerate}

Ebben a szakaszban mindkét lehetséges fejlesztési irány eredményeirõl szeretnék röviden beszámolni a témában fellelhetõ szakirodalom felhasználásával.

\bigskip

Az elsõ irányba mutató fejlesztések közül néhánnyal már találkozhattunk jelen dolgozat keretein belül is. Az \sectref{korok_keresese} alszakaszban elõkerült a gradiensinformáció felhasználása és az elõzetes tudás alapján történõ megfelelõ korlátok bevezetése a számítás gyorsítása végett. Ebben a témában mindenképp megemlítendõ még a kernel alapú Hough-transzformáció.

A \textbf{kernel alapú megvalósítás} lehetõségét a szerzõk -- Fernandes és Oliveira -- 2008-as cikkükben \cite{hough_fernandes} vetették fel, tehát a módszer meglehetõsen friss. Állításuk szerint valósidejû mûködés érhetõ el viszonylag nagy képek esetén is a szavazási protokoll továbbfejlesztésének köszönhetõen. A módszer egyeneskereséskor az $ (r, \theta) $ paraméterezésen nem változtat, de nem önmagukban álló pontokat, hanem körülbelül egy irányban álló pixelek klasztereit (csoportjait) veszi figyelembe. Minden klaszterben egy megfelelõ irányba forgatott elliptikus Gauss-kernellel végzett konvolúció hivatott a csoportra vonatkozó bizonytalanságot modellezni. Ez a megközelítés nemcsak hogy a szavazatok számának csökkentésével képes az algoritmus sebességét növelni, hanem a protokollnak köszönhetõen sokkal tisztább akkumulátort is eredményez. A háttérzajtól mentes tömbben pedig jóval könnyebben tudjuk a lokális maximumokat detektálni. Ezzel összességében az eddigieknél gyorsabb és pontosabb egyenesdetektálást tudunk kivitelezni.

\bigskip

Felismerhetõ formák tekintetében a módszer bõvíthetõ az általános analitikus 2D görbék mellett 3D képeken testek felismerésére. Ezen kívül semmiképp nem hagyhatjuk szó nélkül a Ballard-féle általánosított Hough-transzformációt sem.

Három dimenzióban a \textbf{testek felismerése} a két dimenziós esetekhez hasonlóan történik -- hasonló buktatókkal és legtöbbször további járulékos memória- és számításigénnyel. Síkoknál például az egyenesekhez hasonlóan a megszokott reprezentáció nem használható függõleges síkok keresésére. A megoldást ebben az esetben a gömbi koordinátarendszerre való átállás jelenti, ahol a sík már normálvektorával és az origótól való távolságával jellemezhetõ (a paramétertér tehát három dimenziósra adódik). Gömbök, hengerek, ellipszoidok keresése szintén a két dimenziós esetek tapasztalatainak felhasználásával történhet. Az új, akár térbeli formák keresése kombinálható az eddig megismert teljesítménynövelõ eljárásokkal: a már megismer eljárások használhatók a keresés során, természetesen adott esetben három dimenzióra vonatkoztatva azokat.

Dana H. Ballard általánosított transzformációja \cite{hough_ballard} -- amint az már jelen dolgozat keretein belül is elõkerült -- lehetõvé teszi modelljükkel reprezentált általános objektumok felismerését a \textbf{mintafelismerés} alapelvét kihasználva. A modell elõfordulásainak keresése a képtérben -- ahogy Ballard megmutatta -- szintén megoldható paraméterek akkumulálásával. Válasszuk ezen paramétereknek a modellt a képtérbe átvivõ transzformáció (vagy transzformációk) paramétereit! Például ha csak a modell pozícióját keressük, akkor mindössze eltolási paramétereket kell figyelembe vennünk, ezek akkumulálásával a megszokott módon és eredményességgel ismerhetjük fel az adott objektum helyzetét. Azonban a módszer a transzláció (eltolás) invariancia mellett rotáció (forgatás) és skálázás (méret) invariánssá is tehetõ, igaz a paramétertér dimenziószámának növelése árán. Láthatjuk, hogy a mintafelismerés és a szavazási mechanizmus elõnyeinek ötvözésével ez az eljárás igazán komoly fegyver lehet a képfeldolgozási eszköztárunkban.

\begin{figure}[!ht]
\centering
\includegraphics[width=67mm, keepaspectratio]{figures/generalhough_money_1.png}\hspace{1cm}
\includegraphics[width=67mm, keepaspectratio]{figures/generalhough_money_2.png}
\caption{Kép- és paramétertér az általánosított Hough-transzformáció használata közben.\\Forrás: \url{http://href.hu/x/c91m}}
\label{fig:generalhough}
\end{figure}

%............................................................................
\subsection{Összefoglalás}\label{sect:hough_osszefoglalas}
%............................................................................

Az eddigiek alapján látszik, hogy a transzformáció felhasználhatóságának szûk keresztmetszetét leginkább a nagy számításigénye adja. Ez egyben legfõbb \textbf{hátránya} is: a számításigény a paramétertér dimenziószámának, vagy a kép méretének növekedésével egyre csak növekszik. A dimenziók száma tehát korlátot ad a transzformáció gyakorlati felhasználhatóságának, ezért eredeti formájában leginkább csak egyenesek, vagy körök keresésére használják. A körök keresésének lehetõsége azonban kielégítheti a pupillakövetési probléma által támasztott követelményeket. A követés során \textbf{elõnyére} válhat, hogy a szavazási mechanizmus használata miatt meglehetõsen zajos képeken is képes lehet a pupillakontúr megbízható felismerésére. Nem mesterségesen elõállított tesztesetekben -- hiszen pupillakövetés esetén csak ennek van értelme -- pedig ez korántsem elhanyagolható szempont.


%,,,,,,,,,,,,,,,,,,,,,,,,,,,,,,,,,,,,,,,,,,,,,,,,,,,,,,,,,,,,,,,,,,,,,,,,,,,,
\section{Objektumdetektálás és -követés}\label{sect:objdetect}
%,,,,,,,,,,,,,,,,,,,,,,,,,,,,,,,,,,,,,,,,,,,,,,,,,,,,,,,,,,,,,,,,,,,,,,,,,,,,

%............................................................................
\subsection{A Viola--Jones objektumdetektor}\label{sect:viola}
%............................................................................

A Paul Viola és Michael Jones által 2001-ben elővezetett \cite{vj} Viola--Jones objektumdetektor az első olyan objektumfelismerő rendszer, megközelítheti, vagy akár teljesítheti is a valósidejű feldolgozás által támasztott követelményeket. A rendszer eleinte arcdetektálás céljából készült, de mint kiderült, működése könnyen általánosítható, így megfelelő tanítás után bármilyen objektum felismerésére képes lehet.

\begin{figure}[!ht]
\centering
\includegraphics[width=50mm, keepaspectratio]{figures/features.png}
\caption{A Viola és Jones által használt jellemzőtípusok.\\Forrás: \cite{vj}}
\label{fig:features}
\end{figure}

Az osztályozó úgynevezett \emph{jellemzőkkel} (features) dolgozik, amelyek téglalap alakú területeket alapul véve a bennük lévő képpontok összegét jelentik. Ebben a formában a jellemzők rokonságot mutatnak a Haar bázisfüggvényekkel, amelyek az objektumdetektálásban korábban meghatározó \emph{wavelet transzformációnál} használatosak. Lehetséges jellemzőtípusokat mutat be a \figref{features} ábra.

A jellemző-alapú, úgynevezett \emph{integrális} képreprezentáció esetén az egyes jellemzők ugyan nagyon egyszerűek, viszont konstans időben kiértékelhetők. A jellemzők gyors értékelhetősége azonban nem kompenzálja nagy számukat. Minden egyes területre minden jellemzőt kiértékelve csökkenne a rendszer használatával nyert előny. Ennek kiküszöbölésére a betanítási fázisban a Viola--Jones detektor az AdaBoost \cite{freund} eljárás egy módosított változatát alkalmazza, amely segít kiválasztani a leginkább fontos jellemzőket, valamint korlátokat ad a kész osztályozó általánosító képességére.

Ha csak az AdaBoost eljárás által kiválasztott fontos jellemzőket értékeljük ki, a rendszer alapesetben akkor sem képes a valós idejű osztályozásra. Ennek kivédésére a tanítási folyamatban az egyes elemi osztályozókat egymás után kötik (kaszkádosítják), és minden osztályozó bemenetére csak azok a minták jutnak, amelyeket a sorban előtte lévő összes osztályozó elfogadott.

\bigskip

A fent felsorolt eljárásoknak köszönhetően a Viola--Jones objektumdetektor képes az elődeinél jóval gyorsabban osztályozni a bemenetére küldött képeket. Ha azonban nem is lehetséges a videofolyam minden egyes képkockáján a rendelkezésre álló időn belül önállóan döntést hozni, valamilyen kiegészítő technika használatával (pl. a felismert objektumok optikai áramlás-alapú követése) kellően \emph{robusztus} és \emph{valós idejű} objektumdetektáló rendszer létrehozása válik lehetővé.

%............................................................................
\subsection{A Lucas--Kanade optikai áramlás eljárás}\label{sect:optflow}
%............................................................................

A számítógépes látás területén az optikai áramlás közelítésének kérdése fontos kutatási terület, ennek megfelelően a problémára számos különböző megoldás született. A Bruce D. Lucas és Takeo Kanade által 1981-ben nyilvánosságra hozott \cite{lk} differenciális algoritmus manapság az egyik legszélesebb körben használt módszer optikai áramlás számításakor.

\begin{figure}[!ht]
\centering
\includegraphics[width=80mm, keepaspectratio]{figures/opticalflow.png}
\caption{A kép jobb oldalán piros nyilak jelképezik az optikai áramlást.\\Forrás: \url{http://href.hu/x/dz36}}
\label{fig:optical_flow}
\end{figure}

Az eljárás feltételezi, hogy az áramlás konstans a vizsgált képpont helyi környezetében, és ebben a környezetben (ablakban) a legkisebb négyzetek módszerével próbálja kiszámolni az aktuális pont elmozdulásvektorát. Azzal, hogy nem egy-egy képpontot, hanem egy kis lokális ablakot vesz figyelembe a számítás során, a Lucas--Kanade eljárás kevésbé érzékenyen reagál a zajos képekre. Másrészt viszont a lokális tulajdonság hatására az eljárás nem tud információt szolgáltatni a kép nagy, összefüggő, azonos intenzitású területein.

%. . . . . . . . . . . . . . . . . . . . . . . . . . . . . . . . . . . . . .
\subsubsection{Matematikai háttér}\label{sect:matrixok}
%. . . . . . . . . . . . . . . . . . . . . . . . . . . . . . . . . . . . . .

Tegyük fel, hogy két egymás követő időpillanatban a képek közti eltérés megfelelően kicsi. Az optikai áramlás eljárás alapegyenlete ebben az esetben következő

\begin{align}\label{eq:optflow_alap}
I_x V_x + I_y V_y = - I_t
\end{align}

formában írható fel, ahol $V_x$ és $V_y$ az $(x,y,t)$ képpont sebességének (vagyis optikai áramlásának) $x$ és $y$ koordinátái, $I_x$, $I_y$ és $I_t$ pedig a kép parciális deriváltjai.

A Lucas--Kanade eljárásban egy, a vizsgált képpont körüli ablak tartalmát vesszük figyelembe, és feltesszük, hogy az áramlás ezen a régión belül állandó. A \eqref{optflow_alap} egyenletet ennek megfelelően a következő formára hozhatjuk

\begin{align}\label{eq:optflow_lk}
I_x(q_1)V_x + I_y(q_1)V_y &= - I_t(q_1) \nonumber \\
I_x(q_2)V_x + I_y(q_2)V_y &= - I_t(q_2) \nonumber \\
&\vdots \nonumber \\
I_x(q_n)V_x + I_y(q_n)V_y &= - I_t(q_n)
\end{align}

amelyben $q_1$, $q_2$, $\ldots$, $q_n$ az aktuálisan vizsgált pont körüli ablak egyes képpontjai.

A \eqref{optflow_lk} egyenletet $Av = b$ formában, mátrixokkal is felírhatjuk, ekkor 

\begin{align}\label{eq:optflow_matrix}
A &= \left[ \begin{array}{cc} I_x(q_1) & I_y(q_1) \\ I_x(q_2) & I_y(q_2) \\ \vdots & \vdots \\ I_x(q_n) & I_y(q_n) \end{array} \right] \nonumber \\
v &= \left[ \begin{array}{c} V_x \\ V_y \end{array} \right] \nonumber \\
b &= \left[ \begin{array}{c} -I_t(q_1) \\ -I_t(q_2) \\ \vdots \\ -I_t(q_n) \end{array} \right]
\end{align}

A \eqref{optflow_matrix}-ban definiált egyenletrendszer felülhatározott, vagyis több egyenlet szerepel benne, mint ismeretlen. A Lucas--Kanade algoritmus a legkisebb négyzetek módszerével számítja ki a sebességkomponenseket, vagyis

\begin{align}\label{eq:optflow_seb1}
v &= \frac{A^T b}{A^T A}
\end{align}

ami \eqref{optflow_matrix}-at behelyettesítve a következő alakra hozható

\begin{align}\label{eq:optflow_seb2}
\left[ \begin{array}{c} V_x \\ V_y \end{array} \right] &= \left[ \begin{array}{cc} \sum_{i=1}^n I_x(q_i)^2 & \sum_{i=1}^n I_x(q_i)I_y(q_i) \\ \sum_{i=1}^n I_x(q_i)I_y(q_i) & \sum_{i=1}^n I_y(q_i)^2 \end{array} \right]^{-1} \left[ \begin{array}{c} - \sum_{i=1}^n I_x(q_i)I_t(q_i) \\ - \sum_{i=1}^n I_y(q_i)I_t(q_i) \end{array} \right]
\end{align}

vagyis az aktuálisan vizsgált $q_1$, $q_2$, $\ldots$, $q_n$ pontokat tartalmazó ablak $p$ középpontjának $x$ és $y$ irányban vett sebességkomponenseit adja meg közvetlenül a \eqref{optflow_seb2} egyenlet.


%,,,,,,,,,,,,,,,,,,,,,,,,,,,,,,,,,,,,,,,,,,,,,,,,,,,,,,,,,,,,,,,,,,,,,,,,,,,,
\section{BLOB műveletek}\label{sect:blob}
%,,,,,,,,,,,,,,,,,,,,,,,,,,,,,,,,,,,,,,,,,,,,,,,,,,,,,,,,,,,,,,,,,,,,,,,,,,,,

+++ BLOB-alapú keresés elmélete +++
%----------------------------------------------------------------------------
\chapter{Technológia}\label{sect:technológia}
%----------------------------------------------------------------------------

+++ bevezeto a fejezethez +++

%,,,,,,,,,,,,,,,,,,,,,,,,,,,,,,,,,,,,,,,,,,,,,,,,,,,,,,,,,,,,,,,,,,,,,,,,,,,,
\section{Az OpenCV könyvtár}\label{sect:opencv}
%,,,,,,,,,,,,,,,,,,,,,,,,,,,,,,,,,,,,,,,,,,,,,,,,,,,,,,,,,,,,,,,,,,,,,,,,,,,,

Az OpenCV\footnote{\url{http://opencv.willowgarage.com/}} egy nyílt forráskódú gépi látás (computer vision) könyvtár. Elsődleges célja, keretet nyújtani \textbf{valósidejű} képfeldolgozási alkalmazások fejlesztésére. A könyvtár szabadon letölthető és felhasználható a BSD licenc\footnote{\url{http://www.linfo.org/bsdlicense.html}} keretein belül.

\bigskip

Az OpenCV projekt hivatalosan 1999-ben indult az Intel kezdeményezésében. A nagyközönségnek a 2000. évi \textit{,,IEEE Conference on Computer Vision and Pattern Recognition''} konferencián mutatkozott be, majd öt béta-verziót követően 2006-ban jutott el az 1.0-ás hivatalos kiadásig. A fejlesztése itt úgy tűnt, hogy megáll, de végül a projektet a Willow Garage\footnote{\url{http://www.willowgarage.com/}} robotikai kutatólabor 2008-ban szárnyai alá vette, és azóta is aktív fejlesztés alatt áll. 2008 októberében az 1.1-es verzióval közel egy időben látott napvilágot az elsõ hivatalos OpenCV-vel foglalkozó könyv \textit{,,Learning OpenCV: Computer Vision with the OpenCV Library''} címmel Gary Bradski és Adrian Kaehler fejlesztõk tollából \cite{opencv_book}. Az egy évvel később, 2009 októberében megjelent 2.0-ás verzióval a projekt nagy fejlődésen esett át. Ebben a verzióban található meg elõször a C++ és Python interfész (ez a meglévő C mellett már három hivatalosan fejlesztett interfészt jelent), amely az egyszerûbb kezelhetõség, új függvények mellett a meglévő eljárások teljesítmény tekintetében -- különösen többmagos rendszereken -- jobb implementációját kínálja a felhasználóknak.

\bigskip

Az OpenCV jelenlegi 2.1-es verziója elérhető FreeBSD, Linux, Mac OS és Windows operációs rendszerek alá. Széles körben, mondhatni világszerte használt, felhasználói tábora több, mint 40\,000 főt számlál. Köszönhető ez többek között annak, hogy felhasználási lehetőségei igencsak sokrétűek: több, mint 500 optimalizált algoritmust kínál annak érdekében, hogy ,,ne kelljen újra feltalálnunk a kereket''. Sebesség tekintetében érződik a kipróbált, optimalizált algoritmusok használata: az OpenCV a jelenleg elérhető leggyorsabb alternatíva gépi látás terén (\figref{opencv_speed} ábra). Teljesítménye azonban adott esetben még tovább növelhető, mivel ha Intel IPP\footnote{\url{http://software.intel.com/en-us/intel-ipp/}} (Integrated Performance Primitives) támogatást észlel, az abban található szálakra optimalizált algoritmusok használatát fogja preferálni.

\begin{figure}[!ht]
\centering
\includegraphics[width=100mm, keepaspectratio]{figures/opencv_speed.png}
\caption{Az OpenCV teljesítménye az LTI és VXL képfeldolgozási könyvtárakkal összehasonlítva.\\Forrás: \cite{opencv_book}}
\label{fig:opencv_speed}
\end{figure}

A teljes funkcionalitás részletekbe menõ bemutatása -- mint láthattuk \cite{opencv_book} -- egy könyvet is megtölt, de a teljesség igénye nélkül tekintsük át, hogy milyen alapvető jellemzői és alkalmazásai vannak a környezetnek:

\begin{itemize}

 \item alap adatstruktúrák
  \begin{itemize}
   \item mátrixok, vektorok
  \end{itemize} 

 \item mátrix és vektor manipuláció, lineáris algebra
  
 \item dinamikus adatstruktúrák
  \begin{itemize}
   \item listák, sorok, halmazok
   \item gráfok és fák
  \end{itemize}

 \item kép és videó input/output
  \begin{itemize}
   \item beolvasás fájlból (kép vagy videó) és kameráról
   \item kiírási lehetőség képként vagy videóként
  \end{itemize}

 \item előfeldolgozás
  \begin{itemize}
   \item él- és sarokkeresés
   \item mintavételezés és interpoláció
   \item színkonverzió
   \item morfológiai operátorok
  \end{itemize}

 \item struktúraanalízis
  \begin{itemize}
   \item távolság- és Hough-transzformáció
   \item kontúrfeldolgozás
   \item sablonillesztés
   \item különböző momentumok
   \item Delaunay háromszögelés
  \end{itemize}

 \item kamerakalibráció
  \begin{itemize}
   \item kalibrációs mintázatok felismerése és követése
   \item fundamentális mátrix becslés
   \item homográfia becslés
   \item sztereó megfeleltetés
  \end{itemize}
  
 \item mozgásanalízis
  \begin{itemize}
   \item optical flow
   \item mozgásszegmentálás és -követés
  \end{itemize}

 \item objektumfelismerés
  \begin{itemize}
   \item eigen-módszerek
   \item rejtett Markov-modell (Hidden Markov Model -- HMM)
  \end{itemize}
  
 \item GUI és rajzolás
  \begin{itemize}
   \item kép és videó megjelenítés
   \item billentyűzet és egérkezelés
   \item egyenes, kör, poligon, szöveg rajzolása
  \end{itemize}
  
\end{itemize}

Láthatjuk, hogy a fent felsorolt funkciókkal a gépi látás terén rengeteg egyszerűbb feladatot szinte ,,egy lépésben'', beépített, optimalizált eljárások segítségével oldhatunk meg. Ha nagyobb szabású projektbe kezdünk, akkor is hasznunkra lehet, hogy részben vagy egészében egy több tízezres felhasználói tábor (melynek jelentős részét aktív kutatók alkotják) visszajelzései alapján fejlesztett környezetre építhetjük munkánkat.

Az OpenCV számos előnye mellett hátrányként említhető meg, hogy segítségével a felhasználói felületet csak nagyon leegyszerűsített módon szabhatjuk testre. Igaz, hogy a könyvtár feladata elsősorban a képfeldolgozás és nem a megjelenítés, így ebből a nézőpontból a beépített kép és videó megjelenítési lehetőség, billentyűzet- és egérkezelés valamint trackbarok (csúsztatható kezelőszerv értékek beállítására) létrehozásának lehetősége inkább hozzáadott értékként jelenik meg. Azonban ez nem változtat azon a tényen, hogy ha igény van felhasználóbarát kezelőfelület készítésére, az OpenCV-t integrálnunk kell valamely elterjedt grafikus felhasználói felület toolkittel.

\bigskip

+++ uj funkciok, Qt felulet, atnezni +++

%,,,,,,,,,,,,,,,,,,,,,,,,,,,,,,,,,,,,,,,,,,,,,,,,,,,,,,,,,,,,,,,,,,,,,,,,,,,,
\section{A Qt keretrendszer}\label{sect:qt}
%,,,,,,,,,,,,,,,,,,,,,,,,,,,,,,,,,,,,,,,,,,,,,,,,,,,,,,,,,,,,,,,,,,,,,,,,,,,,

+++ Qt-rol osszefoglalo, Qt creator, screenshot, nyelvek +++

%,,,,,,,,,,,,,,,,,,,,,,,,,,,,,,,,,,,,,,,,,,,,,,,,,,,,,,,,,,,,,,,,,,,,,,,,,,,,
\section{Módosított kamera}\label{sect:infracam}
%,,,,,,,,,,,,,,,,,,,,,,,,,,,,,,,,,,,,,,,,,,,,,,,,,,,,,,,,,,,,,,,,,,,,,,,,,,,,

+++ kamera infok, LED csere, fenykep +++
%----------------------------------------------------------------------------
\chapter{Megvalósítás}\label{sect:hardver}
%----------------------------------------------------------------------------

A rendszer szoftverét C++ nyelven fejlesztettem, az OpenCV gépi látás könyvtár, illetve a Qt felhasználói toolkit segítségével. A feladat megoldása során törekedtem az objektum-orientált paradigma szem előtt tartására, valamint hogy az elkészült rendszer a lehető legkönnyebben testre szabható, illetve igény esetén kiegészíthető legyen.

%Szöveg.

%,,,,,,,,,,,,,,,,,,,,,,,,,,,,,,,,,,,,,,,,,,,,,,,,,,,,,,,,,,,,,,,,,,,,,,,,,,,,
\section{Az OpenCV könyvtár}\label{sect:opencv}
%,,,,,,,,,,,,,,,,,,,,,,,,,,,,,,,,,,,,,,,,,,,,,,,,,,,,,,,,,,,,,,,,,,,,,,,,,,,,

Az OpenCV\footnote{\url{http://opencv.willowgarage.com/}} egy nyílt forráskódú gépi látás (computer vision) könyvtár. Elsődleges célja, keretet nyújtani \textbf{valósidejű} képfeldolgozási alkalmazások fejlesztésére. A könyvtár szabadon letölthető és felhasználható a BSD licenc\footnote{\url{http://www.linfo.org/bsdlicense.html}} keretein belül.

\bigskip

Az OpenCV projekt hivatalosan 1999-ben indult az Intel kezdeményezésében. A nagyközönségnek a 2000. évi \textit{,,IEEE Conference on Computer Vision and Pattern Recognition''} konferencián mutatkozott be, majd öt béta-verziót követően 2006-ban jutott el az 1.0-ás hivatalos kiadásig. A fejlesztése itt úgy tűnt, hogy megáll, de végül a projektet a Willow Garage\footnote{\url{http://www.willowgarage.com/}} robotikai kutatólabor 2008-ban szárnyai alá vette, és azóta is aktív fejlesztés alatt áll. 2008 októberében az 1.1-es verzióval közel egy időben látott napvilágot az elsõ hivatalos OpenCV-vel foglalkozó könyv \textit{,,Learning OpenCV: Computer Vision with the OpenCV Library''} címmel Gary Bradski és Adrian Kaehler fejlesztõk tollából \cite{opencv_book}. Az egy évvel később, 2009 októberében megjelent 2.0-ás verzióval a projekt nagy fejlődésen esett át. Ebben a verzióban található meg elõször a C++ és Python interfész (ez a meglévő C mellett már három hivatalosan fejlesztett interfészt jelent), amely az egyszerûbb kezelhetõség, új függvények mellett a meglévő eljárások teljesítmény tekintetében -- különösen többmagos rendszereken -- jobb implementációját kínálja a felhasználóknak.

\bigskip

Az OpenCV jelenlegi 2.1-es verziója elérhető FreeBSD, Linux, Mac OS és Windows operációs rendszerek alá. Széles körben, mondhatni világszerte használt, felhasználói tábora több, mint 40\,000 főt számlál. Köszönhető ez többek között annak, hogy felhasználási lehetőségei igencsak sokrétűek: több, mint 500 optimalizált algoritmust kínál annak érdekében, hogy ,,ne kelljen újra feltalálnunk a kereket''. Sebesség tekintetében érződik a kipróbált, optimalizált algoritmusok használata: az OpenCV a jelenleg elérhető leggyorsabb alternatíva gépi látás terén (\figref{opencv_speed} ábra). Teljesítménye azonban adott esetben még tovább növelhető, mivel ha Intel IPP\footnote{\url{http://software.intel.com/en-us/intel-ipp/}} (Integrated Performance Primitives) támogatást észlel, az abban található szálakra optimalizált algoritmusok használatát fogja preferálni.

\begin{figure}[!ht]
\centering
\includegraphics[width=100mm, keepaspectratio]{figures/opencv_speed.png}
\caption{Az OpenCV teljesítménye az LTI és VXL képfeldolgozási könyvtárakkal összehasonlítva.\\Forrás: \cite{opencv_book}}
\label{fig:opencv_speed}
\end{figure}

A teljes funkcionalitás részletekbe menõ bemutatása -- mint láthattuk \cite{opencv_book} -- egy könyvet is megtölt, de a teljesség igénye nélkül tekintsük át, hogy milyen alapvető jellemzői és alkalmazásai vannak a környezetnek:

\begin{itemize}

 \item alap adatstruktúrák
  \begin{itemize}
   \item mátrixok, vektorok
  \end{itemize} 

 \item mátrix és vektor manipuláció, lineáris algebra
  
 \item dinamikus adatstruktúrák
  \begin{itemize}
   \item listák, sorok, halmazok
   \item gráfok és fák
  \end{itemize}

 \item kép és videó input/output
  \begin{itemize}
   \item beolvasás fájlból (kép vagy videó) és kameráról
   \item kiírási lehetőség képként vagy videóként
  \end{itemize}

 \item előfeldolgozás
  \begin{itemize}
   \item él- és sarokkeresés
   \item mintavételezés és interpoláció
   \item színkonverzió
   \item morfológiai operátorok
  \end{itemize}

 \item struktúraanalízis
  \begin{itemize}
   \item távolság- és Hough-transzformáció
   \item kontúrfeldolgozás
   \item sablonillesztés
   \item különböző momentumok
   \item Delaunay háromszögelés
  \end{itemize}

 \item kamerakalibráció
  \begin{itemize}
   \item kalibrációs mintázatok felismerése és követése
   \item fundamentális mátrix becslés
   \item homográfia becslés
   \item sztereó megfeleltetés
  \end{itemize}
  
 \item mozgásanalízis
  \begin{itemize}
   \item optical flow
   \item mozgásszegmentálás és -követés
  \end{itemize}

 \item objektumfelismerés
  \begin{itemize}
   \item eigen-módszerek
   \item rejtett Markov-modell (Hidden Markov Model -- HMM)
  \end{itemize}
  
 \item GUI és rajzolás
  \begin{itemize}
   \item kép és videó megjelenítés
   \item billentyűzet és egérkezelés
   \item egyenes, kör, poligon, szöveg rajzolása
  \end{itemize}
  
\end{itemize}

Láthatjuk, hogy a fent felsorolt funkciókkal a gépi látás terén rengeteg egyszerűbb feladatot szinte ,,egy lépésben'', beépített, optimalizált eljárások segítségével oldhatunk meg. Ha nagyobb szabású projektbe kezdünk, akkor is hasznunkra lehet, hogy részben vagy egészében egy több tízezres felhasználói tábor (melynek jelentős részét aktív kutatók alkotják) visszajelzései alapján fejlesztett környezetre építhetjük munkánkat.

Az OpenCV számos előnye mellett hátrányként említhető meg, hogy segítségével a felhasználói felületet csak nagyon leegyszerűsített módon szabhatjuk testre. Igaz, hogy a könyvtár feladata elsősorban a képfeldolgozás és nem a megjelenítés, így ebből a nézőpontból a beépített kép és videó megjelenítési lehetőség, billentyűzet- és egérkezelés valamint trackbarok (csúsztatható kezelőszerv értékek beállítására) létrehozásának lehetősége inkább hozzáadott értékként jelenik meg. Azonban ez nem változtat azon a tényen, hogy ha igény van felhasználóbarát kezelőfelület készítésére, az OpenCV-t integrálnunk kell valamely elterjedt grafikus felhasználói felület toolkittel.

%,,,,,,,,,,,,,,,,,,,,,,,,,,,,,,,,,,,,,,,,,,,,,,,,,,,,,,,,,,,,,,,,,,,,,,,,,,,,
\section{Pszichológiai bemutató}\label{sect:pszicho}
%,,,,,,,,,,,,,,,,,,,,,,,,,,,,,,,,,,,,,,,,,,,,,,,,,,,,,,,,,,,,,,,,,,,,,,,,,,,,

A rendszer működését úgy demonstráltam, hogy végrehajtottam Alfred L. Yarbus orosz pszichológus 1967-es tanulmánya egy részletét. A kísérletben a kutatók azt bizonyították be, hogy a tesztalanyoknak előzetesen különböző kérdéseket feltéve, a kérdések jelentősen befolyásolták egy kép részleteinek vizsgálatát ahhoz képest, ha csak ,,szabadon'' nézegették azt. A teszteredményeket összefoglaló kép -- mint az egyik első a témával foglalkozó eredmény -- jól ismert a tekintetkövetéssel foglalkozók körében, ez látható a \figref{yarbus} ábrán.

\begin{figure}[!ht]
\centering
\includegraphics[width=140mm, keepaspectratio]{figures/yarbus.jpg}
\caption{Yarbus '67-es kísérletének eredménye}
\label{fig:yarbus}
\end{figure}

Látható, hogy a kísérlet végrehajtásához szükség volt a tekintet követésére. Ekkor még a kornak megfelelően nem álltak rendelkezésre kifinomult módszerek: az alanyokat egy meglehetősen kényelmetlen acélszerkezethez rögzítve vizsgálták. A bemutató alkalmazásomban arra kerestem a választ, hogy lehetséges-e hasonló méréseket elvégezni az általam fejlesztett rendszerrel.

\bigskip

Az általam tesztelt alanyoknak a következő kérdésekre kellett válaszolniuk a tesztkép (Repin: Váratlan utazó) rövid vizsgálata után. A vizsgálatot minden kérdésben 200 beérkezett érvényes mérési pontig folytattam, ez a felhasznált webkamera, illetve a feldolgozási sebesség mellett nagyjából kérdésenként 15--20 másodpercet vett igénybe.

\begin{enumerate}
 \item szabad nézelődés
 \item ,,Milyen anyagi körülmények között él a család?''
 \item ,,Adja meg az egyes szereplők életkorát!''
 \item ,,Mit csinálhattak a szereplők, mielőtt az utazó betoppant?''
 \item ,,Milyen ruhát viselnek a kép szereplői?''
 \item ,,Próbáljon megjegyezni minél több strukturális részletet (személyek, tárgyak pozíciója)!''
 \item ,,Mennyi ideig lehetett távol az utazó a családtól?''
\end{enumerate}

A kérdésekre nem volt jó, vagy rossz válasz, a kísérlet minden alanynál csak a feladatnak megfelelő figyelmi területek változását vizsgálja. A vizsgálatom eredménye a \label{fig:eredmeny} ábrán követhető.

\begin{figure}[!ht]
\centering
\includegraphics[width=140mm, keepaspectratio]{figures/yarbus_eredmeny.png}
\caption{Eredmények a saját rendszerrel}
\label{fig:eredmeny}
\end{figure}

Látható -- bár ez nem volt feltétel -- hogy a mérési eredmények ezen alany esetén meglehetősen jól fedik Yarbus tesztalanyának eredményeit. Az viszont mindenképpen kijelenthető, hogy szignifikáns különbség van az ábra első (szabad nézelődés), valamint többi része között. Nem célom a kísérlet pszichológiai eredményeit értékelni, az viszont bizonyos, hogy a rendszer hasonló kísérletek elvégzését támogatja.

%,,,,,,,,,,,,,,,,,,,,,,,,,,,,,,,,,,,,,,,,,,,,,,,,,,,,,,,,,,,,,,,,,,,,,,,,,,,,
\section{Webergonómiai bemutató}\label{sect:web}
%,,,,,,,,,,,,,,,,,,,,,,,,,,,,,,,,,,,,,,,,,,,,,,,,,,,,,,,,,,,,,,,,,,,,,,,,,,,,

Ahogyan dolgozatom első fejezetében már említettem, a tekintet követése a webergonómia területén is fontos kísérletek elvégzésére ad lehetőséget. A rendszer ilyen irányú képességeit demonstrálandó, elkészítettem az pszichológiai kísérlet adatait hőtérképen (heat map) megjelenítő funkciót is, ennek eredménye egy tesztképre a \label{fig:heatmap} ábrán látható.

\begin{figure}[!ht]
\centering
\includegraphics[width=100mm, keepaspectratio]{figures/heatmap.jpg}
\caption{Hőtérképes megjelenítés a pszichológiai teszt egy adathalmazára (,,Adja meg a szereplők életkorát!'')}
\label{fig:heatmap}
\end{figure}

Minden vizsgálatban célszerű a lehető leginkább informatív megjelenítést választani a rendelkezésre álló adathalmaz alapján. A képen egyre pirosabb színnel jelöltem a kép frekventáltan vizsgált területeit, például egy webergonómiai vizsgálatban az adatok effajta vizualizációja lehet a legcélszerűbb.

%............................................................................
%\subsection{Analóg TV-kamera}\label{sect:analog}
%............................................................................

%,,,,,,,,,,,,,,,,,,,,,,,,,,,,,,,,,,,,,,,,,,,,,,,,,,,,,,,,,,,,,,,,,,,,,,,,,,,,
%\section{Összefoglalás}\label{sect:hardver_osszefoglalas}
%,,,,,,,,,,,,,,,,,,,,,,,,,,,,,,,,,,,,,,,,,,,,,,,,,,,,,,,,,,,,,,,,,,,,,,,,,,,,

%Összefoglalás.
%----------------------------------------------------------------------------
\chapter{Megvalósítás}\label{sect:megvalositas}
%----------------------------------------------------------------------------

\texttt{+++ bevezeto a fejezethez +++}

\bigskip

\texttt{+++ melyik szakaszban mi van +++}

%,,,,,,,,,,,,,,,,,,,,,,,,,,,,,,,,,,,,,,,,,,,,,,,,,,,,,,,,,,,,,,,,,,,,,,,,,,,,
\section{Architektúra}\label{sect:architektura}
%,,,,,,,,,,,,,,,,,,,,,,,,,,,,,,,,,,,,,,,,,,,,,,,,,,,,,,,,,,,,,,,,,,,,,,,,,,,,

\texttt{+++ altalanos felepites, osztalydiagram, stb. +++}

%............................................................................
\subsection{A \texttt{Class1} osztaly}\label{sect:class1}
%............................................................................

\texttt{+++ interfesz, mit csinal, stb +++}

%............................................................................
\subsection{A \texttt{Class2} osztaly}\label{sect:class2}
%............................................................................

\texttt{+++ interfesz, mit csinal, stb +++}

%............................................................................
\subsection{A \texttt{Class3} osztaly}\label{sect:class3}
%............................................................................

\texttt{+++ interfesz, mit csinal, stb +++}

%,,,,,,,,,,,,,,,,,,,,,,,,,,,,,,,,,,,,,,,,,,,,,,,,,,,,,,,,,,,,,,,,,,,,,,,,,,,,
\section{Felhasználói felület}\label{sect:gui}
%,,,,,,,,,,,,,,,,,,,,,,,,,,,,,,,,,,,,,,,,,,,,,,,,,,,,,,,,,,,,,,,,,,,,,,,,,,,,

\texttt{+++ leiras + kepek a felhasznalo feluletrol +++}

%,,,,,,,,,,,,,,,,,,,,,,,,,,,,,,,,,,,,,,,,,,,,,,,,,,,,,,,,,,,,,,,,,,,,,,,,,,,,
\section{Felhasználói dokumentáció}\label{sect:docs}
%,,,,,,,,,,,,,,,,,,,,,,,,,,,,,,,,,,,,,,,,,,,,,,,,,,,,,,,,,,,,,,,,,,,,,,,,,,,,

\texttt{+++ az alkalmazas pontos hasznalata (tobb monitor, stb) +++}

%,,,,,,,,,,,,,,,,,,,,,,,,,,,,,,,,,,,,,,,,,,,,,,,,,,,,,,,,,,,,,,,,,,,,,,,,,,,,
\section{Tesztelés, eredmények}\label{sect:teszteles}
%,,,,,,,,,,,,,,,,,,,,,,,,,,,,,,,,,,,,,,,,,,,,,,,,,,,,,,,,,,,,,,,,,,,,,,,,,,,,

\texttt{+++ tesztelesi dokumentacio (ide mit?) +++}
\include{chapter7}
%----------------------------------------------------------------------------
\chapter*{Értékelés}\label{sect:ertekeles}
%----------------------------------------------------------------------------

\texttt{+++ altalanos ertekeles ide +++}

%----------------------------------------------------------------------------
\chapter*{Köszönetnyilvánítás}
%----------------------------------------------------------------------------

Ezúton szeretném megragadni az alkalmat, hogy köszönetet mondjak mindazoknak, akik a diplomatervem elkészítéséhez hozzájárultak.

\bigskip

Mindenekelőtt köszönetemet fejezném ki konzulensem, Kertész Zsolt kitartó támogatásáért, valamint elméleti és gyakorlati szinten is rendkívül hasznos meglátásaiért.

Továbbá szeretném megköszönni családom tagjainak türelmét és támogatását, különös tekintettel a rendszer tesztelése során tanúsított aktív és szíves részvételükre.

%\phantomsection\markboth{Ábrák jegyzéke}{}\listoffigures\newpage
%\phantomsection\markboth{Táblázatok jegyzéke}{}\listoftables\newpage

\phantomsection\markboth{Irodalomjegyzék}{}\bibliography{mybib}
%\addcontentsline{toc}{chapter}{Irodalomjegyzék}
\bibliographystyle{unsrt}

%----------------------------------------------------------------------------
\appendix
%----------------------------------------------------------------------------
\chapter*{Függelék}
\setcounter{chapter}{6}  % a fofejezet-szamlalo az angol ABC 6. betuje (F) lesz
\setcounter{equation}{0} % a fofejezet-szamlalo az angol ABC 6. betuje (F) lesz
\numberwithin{equation}{section}
\numberwithin{figure}{section}
\numberwithin{lstlisting}{section}
\numberwithin{table}{section}
\setcounter{footnote}{0}

%,,,,,,,,,,,,,,,,,,,,,,,,,,,,,,,,,,,,,,,,,,,,,,,,,,,,,,,,,,,,,,,,,,,,,,,,,,,,
\section{Paraméterezés}\label{sect:parameterezes}
%,,,,,,,,,,,,,,,,,,,,,,,,,,,,,,,,,,,,,,,,,,,,,,,,,,,,,,,,,,,,,,,,,,,,,,,,,,,,

\subsection{Pupillakeresés}

\begin{itemize}
  \item \textbf{küszöbözés} -- küszöbérték: 5
  \item \textbf{kontúrméret} -- terület legalább 2500 egység
  \item \textbf{körkörösség} -- 1,07-nél kisebb érték (1,0 a tökéletes kör)
\end{itemize}

\subsection{Validációs mérések}

A validációs méréseket a következő elrendezésben hajtottam végre:

\begin{itemize}
  \item 1280$\times$1024 képpont felbontású monitor, 37,5$\times$30,5 cm fizikai mérettel
  \item 33 cm magas álltámasz, középpontja a monitor közepétől 48 cm-re
\end{itemize}

%,,,,,,,,,,,,,,,,,,,,,,,,,,,,,,,,,,,,,,,,,,,,,,,,,,,,,,,,,,,,,,,,,,,,,,,,,,,,
\section{Környezet}\label{sect:telepites}
%,,,,,,,,,,,,,,,,,,,,,,,,,,,,,,,,,,,,,,,,,,,,,,,,,,,,,,,,,,,,,,,,,,,,,,,,,,,,

A rendszer fejlesztése a következő szoftververziók felhasználásával történt egy \emph{Microsoft Windows 7} operációs rendszert futtató notebookon (2,53 GHz-es Intel Core i5 processzor, 4 GB rendszermemória).

Azonos, vagy kompatibilis verziók használatával a forráskód (lásd \sectref{mellekletek} függelék) bármely \emph{Qt} és \emph{OpenCV} által támogatott rendszerre fordítható.

Az \emph{OpenCV} könyvtárat a \emph{MinGW} csomagban található eszközökkel kell lefordítani\footnote{hivatalos leírás: \url{http://opencv.willowgarage.com/wiki/WindowsInstallGuide}}, a \emph{Qt} rendszerrel való kompatibilitás miatt.  

\begin{itemize}
  \item \textbf{Qt} -- 4.7.4 (32 bit)
  \item \textbf{Qt Creator} -- 2.4.1 (rev. 8cd370e163)
  \item \textbf{OpenCV} -- 2.3.0
\end{itemize}

\newpage
%,,,,,,,,,,,,,,,,,,,,,,,,,,,,,,,,,,,,,,,,,,,,,,,,,,,,,,,,,,,,,,,,,,,,,,,,,,,,
\section{Kibővített osztálydiagram}\label{sect:osztalydiagram}
%,,,,,,,,,,,,,,,,,,,,,,,,,,,,,,,,,,,,,,,,,,,,,,,,,,,,,,,,,,,,,,,,,,,,,,,,,,,,

\begin{figure}[!ht]
\centering
\includegraphics[width=130mm, keepaspectratio]{figures/class_diagram_aa.png}
\end{figure}


\newpage
%,,,,,,,,,,,,,,,,,,,,,,,,,,,,,,,,,,,,,,,,,,,,,,,,,,,,,,,,,,,,,,,,,,,,,,,,,,,,
\section{Mellékletek}\label{sect:mellekletek}
%,,,,,,,,,,,,,,,,,,,,,,,,,,,,,,,,,,,,,,,,,,,,,,,,,,,,,,,,,,,,,,,,,,,,,,,,,,,,

\subsection{Az Interneten}

A fejlesztés során elkészült alkalmazás forráskódja szabadon hozzáférhető egy \emph{GitHub}\footnote{\url{http://www.github.com/}} projektben, a \textbf{\url{https://github.com/obrien/eyetracker}} címen.

A projekt nyitóoldala hivatkozásokat tartalmaz az elérhető mellékletekhez. 

\begin{itemize}
    \item \textbf{könyvtárak elérhetősége}
    \begin{itemize}
      \item az \textbf{OpenCV} könyvtár megfelelő verziója
      \item a \textbf{Qt} keretrendszer és a \textbf{Qt} Creator megfelelő verziója
    \end{itemize}
    
  \item \textbf{képgaléria}
    \begin{itemize}
      \item válogatás az alkalmazás tesztelése és használata során készített képekből
    \end{itemize}
    
  \item \textbf{videógaléria}
    \begin{itemize}
      \item pupillakövetési algoritmus működésének bemutatása
      \item videók a tekintetkövetés működéséről
    \end{itemize}

  \item \textbf{dokumentáció}
    \begin{itemize}
      \item a diplomatervem szövegének \LaTeX{} nyelvű forráskódja
      \item a diplomatervem PDF formátumban
    \end{itemize}
\end{itemize}

\subsection{CD-ROM-on}

A fent felsorolt online elérhető kiegészítéseket a diplomatervem CD-ROM mellékletén is elérhetővé tettem. A lemez könyvtárszerkezete és tartalma a következő.

\begin{itemize}
  \item \texttt{docs} -- jelen diplomaterv PDF formátumban, és a hozzá tartozó \LaTeX{} forrás
  \item \texttt{libs} -- a könyvtárak telepítőfájljai
  \item \texttt{pictures} -- képgaléria
  \item \texttt{source} -- az alkalmazás forráskódja
  \item \texttt{videos} -- az alkalmazás működését bemutató videók
\end{itemize}

\label{page:last}
\end{document}
